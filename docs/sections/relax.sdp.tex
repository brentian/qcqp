\documentclass[../main]{subfiles}

\begin{document}

\section{Relaxations for General QCQP}

As a convention, we assume data matrices are symmetric, i.e., \(Q, A_i \in \mathcal{S}^n\)

Recall homogeneous QCQP for \(x \in \mathbb R^n\):

\begin{equation}
  \begin{aligned}
    \model{HQCQP} \quad \mathrm{Maximize}\quad & x^{T} Q x                                                       \\
    \text { s.t. }  \quad                      & x^{T} A_{i} x\; (\le, =, \ge)  \; b_{i}, \forall i=1, \ldots, m \\
                                               & 0 \le x\le  1
  \end{aligned}
\end{equation}

And inhomogeneous QCQP,

\begin{equation}\label{eq:inhoqcqp}
  \begin{aligned}
    \model{QCQP} \quad \mathrm{Maximize}\quad & x^TQx +q^T x                                     \\
    \mathrm{s.t.}  \quad                      & x^{T} A_i x  + a_i^Tx   \; (\le, =, \ge)  \; b_i \\
                                              & 0 \le x\le  1
  \end{aligned}
\end{equation}

We now assume a standard form with less-than-or-equal-to constraints. We mark some of the trivial techniques below.
\begin{itemize}
  \item One can always reformulate \eqref{eq:inhoqcqp} into a homogeneous problem by increasing the dimension of variables by \(1\).
        \begin{equation}
          \begin{aligned}
            \textrm { Maximize } \quad & x^{T} Q x  + q^Tx         \\
            =                          & \begin{bmatrix}x^T & t\end{bmatrix}
            \begin{bmatrix} Q   & q / 2\\ q^T /2 & o \end{bmatrix}
            \begin{bmatrix} x \\ t\end{bmatrix}                              \\
            \mathrm{s.t.} \quad        & - 1\le t \le 1
          \end{aligned}
        \end{equation}
  \item Also, a symmetrized version can be achieved by fact that \(x^TAx = x^TA^Tx\), let
        \begin{equation}
          \tilde A := \frac{A+A^T}{2}
        \end{equation}
\end{itemize}




\subsection{SDP Relaxation}
For \(x \in \mathbb{R}^{n}\), we have: \(x^{T} A_{i} x = A_i \bullet (xx^T)\) and \(xx^T \in \mathcal{S}^n_{+}\), which results in following relaxation using semidefinite cones, also called \textit{lifting} method or Shor relaxation,

\begin{equation}\label{eq:shor_basic}
  \begin{aligned}
    \model{Shor-Basic}\quad \mathrm{Maximize}\quad & Q\bullet Y   + q^T x                                             \\
    \mathrm{s.t.}  \quad                           & Y-xx^T \succeq 0 \text { or }\begin{bmatrix} 1 & x^{T} \\ x & Y \end{bmatrix} \succeq 0 \\
                                                   & A_i \bullet Y + a_i^Tx \; \le  \; b_i, \forall i                 \\
                                                   & 0\le x\le 1
  \end{aligned}
\end{equation}

Notice QCQP with matrix variables can also be reformulated into a SDP based problem, let \(X \in \mathbb{R}^{n\times d}\), then \(X^{T} A_{i} X = A_i \bullet (XX^T)\), similarly,
\begin{equation}
  \begin{aligned}
    \mathrm{Maximize}\quad & Q\bullet Y                                                       \\
    \mathrm{s.t.}  \quad   & Y-XX^T \succeq 0 \text { or }\begin{bmatrix} I_d & X^{T} \\ X & Y \end{bmatrix} \succeq 0 \\
                           & A_i \bullet Y \; \le  \; b_i, \forall i                          \\
  \end{aligned}
\end{equation}

SDP relaxations \eqref{eq:shor_basic} can be unbounded in some case. A simple improvement to \eqref{eq:shor_basic} is to add bounds for the diagonal entries.

\begin{equation}\label{eq:shor}
  \begin{aligned}
    \model{Shor} \quad\mathrm{Maximize}\quad & Q\bullet Y   + q^T x                                             \\
    \mathrm{s.t.}  \quad                     & Y-xx^T \succeq 0 \text { or }\begin{bmatrix} 1 & x^{T} \\ x & Y \end{bmatrix} \succeq 0 \\
                                             & A_i \bullet Y + a_i^Tx \; \le  \; b_i, \forall i                 \\
                                             & 0\le x\le 1                                                      \\
                                             & \mathrm{diag}(Y) \le x
  \end{aligned}
\end{equation}


\subsubsection{Dual}
Consider the dual of primal SDP relaxation with diagonal bound, cf. \eqref{eq:shor}
\begin{equation}
  \begin{aligned}
    L = & \min_{x, Y} \; -\; Q\bullet Y - q^Tx + \sum_i \left( \lambda_i A_i \bullet Y  + \lambda_i a_i^Tx - \lambda_i b_i \right ) \\
        & \diag(v) \bullet Y - v^T x + \mu^Tx - \mu^Te                                                                              \\
        & - \begin{bmatrix}Y & x \\ x^T & 1\end{bmatrix} \bullet
    \begin{bmatrix}Z & y \\ y^T & \alpha\end{bmatrix}                                                                                                      \\
    =   & \min_{x, Y} \; - \lambda^T b - \mu^Te - \alpha                                                                            \\
        & + Y \bullet \left(-Q + \sum_i\lambda_iA_i - Z + \diag(v)\right)                                                           \\
        & + x^T\left(- q + \sum_i \lambda_i a_i + \mu - 2 y - v\right)
  \end{aligned}
\end{equation}

And thus the dual \(\phi(\cdot) = \max_{(\cdot)} L\):

\begin{equation}
  \begin{aligned}
    \model{Dual-Shor} \quad \textrm{Minimize: }\quad & \lambda^Tb + \mu^Te + \alpha                   \\
    \mathrm{s.t.} \quad                              & Q = \sum_i \lambda_i A_i  + \diag(v)  -Z       \\
                                                     & \sum_i \lambda_i a_i + \mu - 2 y - v - q \ge 0 \\
                                                     & \begin{bmatrix}Z & y \\ y^T & \alpha\end{bmatrix} \succeq 0           \\
                                                     & \lambda, \mu, v \ge 0
  \end{aligned}
\end{equation}

There are many further enhancements to \eqref{eq:shor}, see \cite{bao_semidefinite_2011} for discussion on the strength of different relaxations.


\end{document}