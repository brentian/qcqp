\documentclass[../main]{subfiles}
\title{Chordal}
\author{Chuwen Zhang}
\date{\today}
\begin{document}
% \subfile{sections/intro.tex}
% \subfile{sections/sdp.tex}
% \subfile{sections/app.tex}
% \subfile{sections/app.qkp.tex}
% \subfile{sections/bc.tex}
% \subfile{sections/chord.tex}


\section{Graph Theory Review}

\subsection{Miscellanea}
\begin{itemize}
      \item Complement of a graph is the graph with same vertices and complement of edges.
            \[G^\star\left(V, V^2 \backslash E\right)\]

\end{itemize}
\subsection{Packing, Covering, et cetera}



\subsubsection*{Packing}
Set packing: finite set \(S\) and a list of subsets of \(S\). Then, the set packing problem asks if some \(k\) subsets in the list are pairwise disjoint, and its optimization determining the maximum \(k\) 。
\begin{itemize}
      \item \textbf{IS} (independent) stable set: set \(S \subseteq V\) is a stable of an
            undirected graph \(G = (V, E)\) if no two vertices in \(S\) are
            adjacent, i.e., the induced edge set is empty, sometimes referred to as ``vertex packing''
      \item The size of the
            largest stable set is called the stable set number of the graph,
            denoted \(\alpha(G)\). The problem \textbf{(MIS) Maximum independent set} denotes the problem of finding the largest such set.
            \begin{equation}
                  \begin{aligned}
                         & E(S) \equiv \{(v, w) \in E | v, w \in S\} \\
                         & \mathscr S = \{S: E(S)= \emptyset\}       \\
                         & \alpha (G) = \max_{S\in \mathscr S} |S|
                  \end{aligned}
            \end{equation}

      \item Vertex coloring: assignment of vertices in a way that no adjacent vertices are of same color, corresponds to a partition of its vertex set into independent subsets. the \textbf{chromatic number} is defined as the minimum needed colors, has the following relation,
            \begin{equation}
                  \chi(G) \ge \frac{|G|}{\alpha(G)}
            \end{equation}
\end{itemize}

\subsubsection*{Covering}
Set covering, formally, given a universe \(\) and a collection of subsets \(\mathscr S = \{S\}\), the set cover problem is to find such cover, i.e., \(\mathscr U \subseteq \bigcup_{S\in\mathscr S} S \)。 Generally, we are interested in the minimum covers.
\begin{itemize}
      \item Vertex cover: a subset \(V' \subseteq V\) such that \(V'\) covers at least one endpoint of edges. It is not strictly a cover since it is not set of sets, minimal size be vertex cover number \(\tau(G)\)
\end{itemize}

\subsubsection*{Clique and Completeness}
\begin{itemize}
      \item[\(\checkmark\)] fun fact: in the social sciences, clique is a group of individuals who interact with one another and share similar interests
      \item Complete (induced subgraph): a (sub)-graph is complete if all vertices are pairwise adjacent.
      \item (Maximal) Clique: a clique is a set of vertices \(C \subseteq V\) that induces a (maximal) complete subgraph. Normally we use two terms interchangeably except that maximal is declared. \textbf{Maximal}: i.e., it cannot be extended by including one more adjacent vertex
      \item Simplicial: A vertex \(v\) is simplicial if its neighborhood \(adj(v)\) is complete.
      \item Clique cover (CC): a clique cover \(\mathscr C\) of \(G\) is a set of
            cliques that cover the vertex set \(V\). The clique cover number
            \(\bar \chi (G)\) is the number of cliques in the cover. \textbf{(MCC) Minimum clique cover} is the CC with minimized number of cliques.

            \begin{equation}
                  \begin{aligned}
                         & V \subseteq \mathscr C \equiv \bigcup_k C_k    \\
                         & \bar \chi (G) = \min_{\mathscr C} |\mathscr C|
                  \end{aligned}
            \end{equation}
            % \item As as comparison, clique edge cover (CEC) is a set of cliques to cover all edges in \(G\)
            % \item Intersection graph respect a cover \(\{S_i\}\), is a graph by representative vertex \(v_i, \forall i\) as vertex set \(V\), and edge set,

            %       \[E(G)=\left\{(v_{i}, v_{j}) \mid i \neq j, S_{i} \cap S_{j} \neq \emptyset \right\}\]

            %       intersection number is the smallest number of elements of the cover: \(|\bigcup_i S_i|\)
\end{itemize}



\subsubsection*{Remark: complementarity}
We mark the relationship and paraphrases here,
\begin{itemize}
      \item A set \(S\) in \(G\) is independent \textbf{iff.} \(V\backslash S\) is a vertex cover (VC), i.e.,
            \[\tau(G) = |V| - \alpha(G)\]
      \item A set in \(G\) is independent \textbf{iff.} it is a clique in the \(G^\star\)
            \textbf{Remark}: finding \(\alpha(G)\) is equivalent to find the \(\chi(G)\).
      \item For a clique cover \(\mathscr C\) and independent set \(S\), we always have, \(|\mathscr C| \ge |S|\) and,
            \[\bar \chi (G) \ge \alpha(G)\]
            \textbf{PF}: \(\forall k, C_k \) covers at most one vertex in \(S\). The clique cover number and independent number is obvious by taking min and max on left and right, respectively.
\end{itemize}

\subsubsection*{Remark: Complexity in the general case}

Generally, finding a maximum independent set (and thus maximum clique) is a strongly NP-hard problem, and its decision version is NP-complete. Also, minimum clique cover also is NP-hard. Most of the results can be found in \cite{schrijver_combinatorial_2003}.


\subsection{Ordering and the Chordal Graph}

\subsubsection*{Chord and Chordal Graph}

An undirected graph \(G\) is called chordal (or \textbf{triangulated}, or a rigid circuit) if every cycle of
length greater than or equal to 4 has at least one chord, an edge connecting two non-adjacent vertices.

\begin{itemize}
      \item non-chordal graphs can always be chordal extended, i.e., extended to a chordal graph, by adding additional edges to the original graph.
\end{itemize}

ordering: an ordering \(\sigma \{1, 2, ... , n\} \Rightarrow V\) can
also be interpreted as a sequence of vertices
\(\sigma = (\sigma(1), ... , \sigma(n))\). We refer to
\(\sigma^{−1} (v)\) as the index of vertex \(v\) of such ordering.

If \(G\) is chordal, we have the following theorem.

\begin{theorem}
      (\cite{gavril_algorithms_1972}, \cite{schrijver_combinatorial_2003}, \cite{vandenberghe_chordal_2015})
      The stable set number is equal to minimum  \(\bar \chi (G) = \alpha(G)\) if \(G\) is chordal,
      polynomial time algorithms exist for both numbers, see \cite{gavril_algorithms_1972}
\end{theorem}


\begin{itemize}
      \item denote a ordered graph as \(G_\sigma\)
\end{itemize}
\bibliography{headers/qcqp}
\bibliographystyle{headers/spmpsci}
\addcontentsline{toc}{section}{References}

% finish off


\end{document}