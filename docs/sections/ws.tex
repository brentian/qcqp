\documentclass[../main]{subfiles}
\title{Warm start}
\author{Chuwen Zhang}
\date{\today}
\begin{document}
\maketitle
{
    \setcounter{tocdepth}{3}
    \tableofcontents
}
\section{Warm start}
Recall primal dual SDP (Shor) relaxation.
\begin{equation}
    \begin{aligned}
        \model{Shor} \quad\mx\quad & Q\bullet X   + q^T x                                             \\
        \mathrm{s.t.}  \quad       & X-xx^T \succeq 0 \text { or }\begin{bmatrix} 1 & x^{T} \\ x & X \end{bmatrix} \succeq 0 \\
                                   & 0\le x\le 1                                                      \\
    \end{aligned}
\end{equation}
For simplicity, we partition (extra) constraints into two subsets.
\begin{align}
    \label{eq.con.diag}    & \mathrm{diag}(X) \le x                           \\
    \label{eq.con.regular} & A_i \bullet X + a_i^Tx \; \le  \; b_i, \forall i
\end{align}
We mark \eqref{eq.con.diag} as diagonal constraint, and \eqref{eq.con.regular} as quadratic constraints that also includes linear constraints.
\begin{equation}
    \begin{aligned}
        \model{Dual-Shor} \quad \mn\quad & \lambda^Tb + \mu^Te + \alpha                   \\
        \mathrm{s.t.} \quad              & Q = \sum_i \lambda_i A_i  + \diag(v)  -Z       \\
                                         & \sum_i \lambda_i a_i + \mu - 2 y - v - q \ge 0 \\
                                         & \begin{bmatrix}Z & y \\ y^T & \alpha\end{bmatrix} \succeq 0            \\
                                         & \lambda, \mu, v \ge 0
    \end{aligned}
\end{equation}


\begin{lemma}
    (Unconstrained) Suppose \(m = 0\), our primal dual formulation reduces to,
    \begin{equation}
        \begin{aligned}
            \model{Shor} \quad\mx\quad & Q\bullet X   + q^T x                                             \\
            \mathrm{s.t.}  \quad       & X-xx^T \succeq 0 \text { or }\begin{bmatrix} 1 & x^{T} \\ x & X \end{bmatrix} \succeq 0 \\
                                       & 0\le x\le 1                                                      \\
                                       & \mathrm{diag}(X) \le x
        \end{aligned}
    \end{equation}
    \begin{equation}
        \begin{aligned}
            \model{Dual-Shor} \quad \mn\quad & \mu^Te + \alpha                     \\
            \mathrm{s.t.} \quad              & Q = \diag(v)  -Z                    \\
                                             & \mu - 2 y - v - q \ge 0             \\
                                             & \begin{bmatrix}Z & y \\ y^T & \alpha\end{bmatrix} \succeq 0 \\
                                             & \mu, v \ge 0
        \end{aligned}
    \end{equation}
\end{lemma}


% \subsection{Linear Cut}

% Suppose we have primal \((x, X)^{k}\) and dual \((\lambda, \mu, v, y, Z, \alpha)^k\) at some iteration \(k\). Assume the following valid inequality for the primal solutions,
% \begin{equation*}
%     B\bullet X + c^T x \le e
% \end{equation*}

% The dual variable for the new problem \(\tilde\lambda\) has the following form,
% \begin{equation}
%     \tilde\lambda \equiv \begin{bmatrix}
%         \lambda \\
%         t
%     \end{bmatrix}
% \end{equation}

% \begin{equation} \label{eq:ws.dual.linearc}
%     \begin{aligned}
%         \mn\quad            & t \cdot e +  \lambda^Tb + \mu^Te + \alpha                  \\
%         \mathrm{s.t.} \quad & Q = t \cdot B + \sum_i \lambda_i A_i  + \diag(v)  - Z      \\
%                             & t \cdot c + \sum_i \lambda_i a_i + \mu - 2 y - v - q \ge 0 \\
%                             & \begin{bmatrix}Z & y \\ y^T & \alpha\end{bmatrix} \succeq 0                        \\
%                             & t, \lambda, \mu, v \ge 0
%     \end{aligned}
% \end{equation}

% We see \eqref{eq:ws.dual.linearc} has a dummy feasible solution such that \((\begin{bmatrix}
%     \lambda^k \\ 0\end{bmatrix}, \mu^k, v^k, y^k, Z^k, \alpha^k)\)

% \paragraph{Left child}
% Consider the left child \(x_i \in [l_i, x_i^k < u_i]\), the lower bound is unchanged
% \begin{itemize}
%     \item Since \(x_i \le x_i^k\) is active, then \(x_i < 1, \mu_i = 0\)
% \end{itemize}

\subsection{SDP based warmstart}

\paragraph{Standard form}
We first reformulate into standard SDP, by adding a slack \(d\) to diagonal constraint \eqref{eq.con.diag}, and \(s\) to the other constraints.


For diagonal constraint,
\begin{align*}
    \diag(X) \le x & \Leftrightarrow e_ie_i^T \bullet X - e_i^T x + d^Te + 0^Ts = 0                                 \\
                   & \Leftrightarrow e_ie_i^T \bullet X - e_i^T x + e_i^T\diag(d)e_i + \bm 0^m \bullet \diag(s) = 0
\end{align*}

To ease our notation, let:
\begin{align*}
     & \tilde X = \begin{bmatrix} X             & x \\ x^T & 1 \end{bmatrix}, \tilde Q = \begin{bmatrix} Q & \frac{q}{2} \\ \frac{q^T}{2} & 0 \end{bmatrix} ; \\
     & \tilde E_i = \begin{bmatrix} e_ie_i^T  & -\frac{e_i}{2} \\ -\frac{e_i^T}{2} & 0 \end{bmatrix}, \forall i=1,\cdots,n;                \\
     & \tilde A_i = \begin{bmatrix} A_i             & \frac{a_i}{2} \\ \frac{a_i^T}{2} & 0 \end{bmatrix}, \forall i=1,\cdots,m;               \\
     & \tilde y = \begin{bmatrix} \alpha; & z; & y \end{bmatrix}
\end{align*}

Using SDP cone and linear cones,
\begin{equation} \label{eq:ws.std.primal}
    \begin{aligned}
        \mx\quad            & \tilde Q \bullet \tilde X                                                         \\
        \mathrm{s.t.} \quad & \tilde X  \in \mathscr S^{n+1}_+, d \in \mathbb{R}^n_+, s
        \in \mathbb{R}^m_+                                                                                      \\
                            & \begin{bmatrix}  \bm 0_n &  \\  & 1  \end{bmatrix} \bullet \tilde X = 1                                   \\
                            & \tilde E_i \bullet \tilde X + d_i = 0                     & \forall i=1,\cdots, n \\
                            & \tilde A_i \bullet \tilde X + s_i = b_i                   & \forall i=1,\cdots, m \\
    \end{aligned}
\end{equation}

And dual form,

\begin{equation} \label{eq:ws.std.dual}
    \begin{aligned}
        \mn \quad           & \alpha + b^Ty                                                                           \\
        \mathrm{s.t.} \quad & \alpha \in \mathbb{R}, y \in \mathbb{R}^m, z\in \mathbb{R}^n, Y  \in \mathscr S^{n+1}_+ \\
                            & Y =\alpha \cdot \begin{bmatrix}  \bm 0_n &  \\  & 1  \end{bmatrix}
        + \sum_{i=1}^{n} z_{i} \tilde{E_i}
        + \sum_{i=1}^{m} y_{i} \tilde{A_i}
        - \tilde{Q}
    \end{aligned}
\end{equation}

The perturbed complementary condition,

\begin{equation} \label{eq:ws.std.lc}
    \begin{aligned}
         & Y = \tilde X = \mu I_{n+1}           \\
         & z = d =  \mu e_n, \; y = s = \mu e_m
    \end{aligned}
\end{equation}

% \paragraph{Primal Dual}
% We construct an initial point from last iterate and an initial point by \eqref{eq:ws.std.lc}.

% \begin{align*}
%     Y := \lambda Y^0 + (1-\lambda) \mu I_{n+1}                \\
%     \tilde X :=  \lambda \tilde X^0 + (1-\lambda) \mu I_{n+1} \\
%     d  :=  \lambda d^0 + (1-\lambda) \mu e_{n}                \\
%     y :=  \lambda y^0 + (1-\lambda) \mu e_{m}                 \\
% \end{align*}

\paragraph{Dual}
The dual scaling algorithm \cite{benson_solving_2000} solves \eqref{eq:ws.std.dual} by the following implementation, let

\begin{equation}
    \mathcal{A}^T \tilde y = \alpha \cdot \begin{bmatrix}  \bm 0_n &  \\  & 1  \end{bmatrix}
    + \sum_{i=1}^{n} z_{i} \tilde{E_i}
    + \sum_{i=1}^{m} y_{i} \tilde{A_i}, \tilde b = [1; 0; b]
\end{equation}

The dual scaling formulation,

\begin{equation} \label{eq:ws.dsdp.dual}
    \begin{aligned}
        % \mn \quad           & C\bullet X+u^{T} x^{u}-l^{T} x^{l}              \\
        % \mathrm{s.t.} \quad & \mathcal{A} X+
        %                     & \langle I, X\rangle                             \\
        %                     & \langle I, X\rangle                             \\
        %                     & X \in K, \quad x^{u} \geq 0, \quad x^{l} \geq 0 \\
        \mx \quad           & \tilde b^T \tilde y-\Gamma r                                              \\
        \mathrm{s.t.} \quad & - \tilde{Q}-\mathcal{A}^T \tilde y+I_{n+1} r= Y, Y \in \mathscr S^{n+1}_+ \\
                            & \quad r \ge 0                                                             \\
                            & \tilde y_{[1:]} \le 0
    \end{aligned}
\end{equation}

\(\Gamma \ge 0\) corresponds to the upper bound over \(\trace(\tilde X)\) by \(I_{n+1} \bullet \tilde X \le \Gamma\) in the primal problem. The initial point \(\tilde y_0, Y_0, r_0\) by \(y_0 = 0, r_0 = \tau\) where \(\tau\) is big enough to make \(Y_0\) psd.
Suppose we add new cuts to the problem, in the dual problem, we have a set of extra dual variables \(\mu\):

\begin{equation}
    \mathcal B^T\mu = \sum_i \mu_i \tilde B_i
\end{equation}



\begin{equation}
    \begin{aligned}
        % \mn \quad           & C\bullet X+u^{T} x^{u}-l^{T} x^{l}              \\
        % \mathrm{s.t.} \quad & \mathcal{A} X+
        %                     & \langle I, X\rangle                             \\
        %                     & \langle I, X\rangle                             \\
        %                     & X \in K, \quad x^{u} \geq 0, \quad x^{l} \geq 0 \\
        \mx \quad           & \tilde b^T \tilde y + \tilde f^T \mu-\Gamma r                                                 \\
        \mathrm{s.t.} \quad & - \tilde{Q}-\mathcal{A}^T \tilde y - \mathcal{B}^T\mu +I_{n+1} r= Y, Y \in \mathscr S^{n+1}_+ \\
                            & \quad r \ge 0                                                                                 \\
                            & \tilde y_{[1:]} \le 0, \mu \le 0
    \end{aligned}
\end{equation}

The initial point requires the tuple \((z_0, \textcolor{red}{\tilde y_0}, Y_0, \textcolor{red}{\mu_0}, \textcolor{red}{r_0})\) satisfy the following condition,
\begin{align}
    z_0     & = - \tilde Q \bullet \tilde X_0                          \\
    Y_0     & = - \tilde Q   - \mathcal A^T\tilde y_0    + I_{n+1} r_0 \\
    \mu_0   & = 0                                                      \\
    \nu_0 I & = Y_0  \tilde X_0                                        \\
    \nu_0   & = (z_0 - b^Ty_0 + \Gamma r_0) / n \rho_n
\end{align}

The initial point requires \(\Gamma\) be reasonably big.

% By last solution \(y_0\), we can find an estimate,

% \begin{equation}
%     a
% \end{equation}

\subsection{RLT after Branching}
We consider the simple case within the branch and cut process. Suppose we pivot on a \(x_i^k\) which lies in the interior of \(\left[l_i + \epsilon, u_i -\epsilon\right]\) with small tolerance \(\epsilon \ge 0\).

The RLT cuts corresponding to bounds,
\begin{align}
    \label{eq:ws.rlt.left} & X_{ij} - u_i x_j - l_j x_i + l_j u_i \le 0        \\
    \label{eq:ws.rlt.mat}  & \tilde X \bullet \begin{bmatrix}
        e_ie_j^T                  & - (l_j e_i + u_i e_j) / 2 \\
        - (l_j e_i + u_i e_j) / 2 & u_i l_j
    \end{bmatrix} \le 0
\end{align}

\subsection{Implementation Details}
\subsubsection{Modeling slack variables}
\paragraph{Pure SDP}
\eqref{eq:ws.std.primal} and \eqref{eq:ws.std.dual} can also be modeled by pure SDP cones.
\begin{equation}
    \begin{aligned}
         & \tilde n = n + 1 + (n + m)                                       \\
         & \tilde{X} = \begin{bmatrix}
            xx^T & x &   &   \\
            x^T  & 1 &   &   \\
                 &   & d &   \\
                 &   &   & s
        \end{bmatrix}\in \mathscr S^{\tilde{n}}
    \end{aligned}
\end{equation}

And the constraints,
\begin{equation}
    x^TA_ix + a_i^Tx + e_i^T\diag(s)e_i = \tilde{X} \bullet    \begin{bmatrix}
        A_i             & \frac{a_i}{2} &         &          \\
        \frac{a_i^T}{2} & 0             &         &          \\
                        &               & \bm 0_n &          \\
                        &               &         & e_ie_i^T
    \end{bmatrix} = b_i
\end{equation}

The standard form, homogeneous SDP,
\begin{equation}
    \begin{aligned}
        \mx\quad            & \begin{bmatrix}
            Q & \frac{q}{2} & 0 \\ \frac{q^T}{2} & 0 & 0 \\ 0 & 0 & \bm 0_{m+n}
        \end{bmatrix} \bullet \tilde X                               \\
        \mathrm{s.t.} \quad & \tilde{X}  \in \mathscr S^{\tilde{n}}_+                                   \\
                            & \begin{bmatrix}  \bm 0_n &  &  \\  & 1 &  \\  & & \bm 0_{m+n} \end{bmatrix} \bullet \tilde X = 1                           \\
                            & \begin{bmatrix}
            e_ie_i^T         & -\frac{e_i}{2} &          &         \\
            -\frac{e_i^T}{2} & 0              &          &         \\
                             &                & e_ie_i^T &         \\
                             &                &          & \bm 0_m
        \end{bmatrix} \bullet \tilde X = 0   & \forall i=1,\cdots, n \\
                            & \begin{bmatrix}
            A_i             & \frac{a_i}{2} &         &          \\
            \frac{a_i^T}{2} & 0             &         &          \\
                            &               & \bm 0_n &          \\
                            &               &         & e_ie_i^T
        \end{bmatrix}\bullet \tilde{X} = b_i & \forall i=1,\cdots, m \\
    \end{aligned}
\end{equation}

\end{document}
