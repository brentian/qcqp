\documentclass[../main]{subfiles}

\begin{document}
\section{Warm start}
Recall primal dual SDP (Shor) relaxation.
\begin{equation}\label{eq:shor}
    \begin{aligned}
        \model{Shor} \quad\mathrm{Maximize}\quad & Q\bullet Y   + q^T x                                             \\
        \mathrm{s.t.}  \quad                     & Y-xx^T \succeq 0 \text { or }\begin{bmatrix} 1 & x^{T} \\ x & Y \end{bmatrix} \succeq 0 \\
                                                 & A_i \bullet Y + a_i^Tx \; \le  \; b_i, \forall i                 \\
                                                 & 0\le x\le 1                                                      \\
                                                 & \mathrm{diag}(Y) \le x
    \end{aligned}
\end{equation}
\begin{equation}
    \begin{aligned}
        \model{Dual-Shor} \quad \textrm{Minimize: }\quad & \lambda^Tb + \mu^Te + \alpha                   \\
        \mathrm{s.t.} \quad                              & Q = \sum_i \lambda_i A_i  + \diag(v)  -Z       \\
                                                         & \sum_i \lambda_i a_i + \mu - 2 y - v - q \ge 0 \\
                                                         & \begin{bmatrix}Z & y \\ y^T & \alpha\end{bmatrix} \succeq 0            \\
                                                         & \lambda, \mu, v \ge 0
    \end{aligned}
\end{equation}
\subsection{Linear Cut}

Suppose we have primal \((x, Y)^{k}\) and dual \((\lambda, \mu, v, y, Z, \alpha)^k\) at some iteration \(k\). We let \(B\bullet Y + c^T x \le e\) be a valid inequality for the primal solutions. The dual variable for the new problem \(\tilde\lambda\) has the following form,
\begin{equation}
    \tilde\lambda \equiv \begin{bmatrix}
        \lambda \\
        t
    \end{bmatrix}
\end{equation}

\begin{equation} \label{eq:ws.dual.linearc}
    \begin{aligned}
        \textrm{Minimize: }\quad & t \cdot e +  \lambda^Tb + \mu^Te + \alpha                  \\
        \mathrm{s.t.} \quad      & Q = t \cdot B + \sum_i \lambda_i A_i  + \diag(v)  - Z      \\
                                 & t \cdot c + \sum_i \lambda_i a_i + \mu - 2 y - v - q \ge 0 \\
                                 & \begin{bmatrix}Z & y \\ y^T & \alpha\end{bmatrix} \succeq 0                        \\
                                 & t, \lambda, \mu, v \ge 0
    \end{aligned}
\end{equation}

We see \eqref{eq:ws.dual.linearc} has a dummy feasible solution such that \((\begin{bmatrix}
    \lambda^k \\ 0\end{bmatrix}, \mu^k, v^k, y^k, Z^k, \alpha^k)\)

\subsubsection{RLT after Branching}
We consider the simple case within the branch and cut process. Suppose we pivot on a \(x_i^k\) which lies in the interior of \(\left[l_i + \epsilon, u_i -\epsilon\right]\) with small tolerance \(\epsilon \ge 0\).

The RLT cuts corresponding to bounds,
\begin{align}
    \label{eq:ws.rlt.left}  & Y_{ij} - u_i x_j - l_j x_i - l_j u_i \le 0 \\
    \label{eq:ws.rlt.right} & Y_{ij} - u_j x_i - l_i x_j - l_i u_j \le 0
\end{align}

Consider the left child \(x_i \in [l_i, x_i^k < u_i]\), the lower bound is unchanged
\begin{itemize}
    \item Since \(x_i \le x_i^k\) is active, then \(x_i < 1, \mu_i = 0\)
\end{itemize}



\end{document}
