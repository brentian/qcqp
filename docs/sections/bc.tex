\documentclass[../main]{subfiles}

\begin{document}
\section{Branch and Cut Algorithm}


for \(x \in \mathbb{R}^{n}\), we have: \(x^{T} A_{i} x = A_i \bullet (xx^T)\)
\begin{equation}
    \begin{aligned}
        \mathrm{Maximize}\quad & Q\bullet Y                                                        \\
        \mathrm{s.t.}  \quad   & Y-xx^T \succeq 0 \text { or } \begin{bmatrix} 1 & x^{T} \\ x & Y \end{bmatrix} \succeq 0 \\
                               & A_i \bullet Y \; (\le, =, \ge) \; b_i, \forall i                  \\
    \end{aligned}
\end{equation}
Possible ways to do this.

\cite{audet_branch_2000} using RLT relaxation and LP, literally, \(W_{ij} \approx x_i x_j , v_i \approx x_i^2\). The branching is essentially based on \(\|W_{ij} - x_ix_j\|, \|v_i - x_i^2\|\), at each node we solve a linear programming relaxation.

For MINLP and more specifically for QP, see \cite{belotti_mixed-integer_2013}, \cite{misener_glomiqo_2013}.

More recently, spacial branch-and-cut method \cite{chen_spatial_2017}.


Some source code to look at:

\begin{itemize}
    \item Couenne: https://www.coin-or.org/Couenne/
    \item
\end{itemize}

\subsection{Root}

The root of the problem can selected from different SDPs as we discussed before. It is a question to answer whether we should use the SDP with the tightest bound.


\end{document}