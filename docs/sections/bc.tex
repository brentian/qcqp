\documentclass[../main]{subfiles}

\begin{document}
\section{Branch and Cut Algorithm}


For \(x \in \mathbb{R}^{n}\), we have: \(x^{T} A_{i} x = A_i \bullet (xx^T)\)
\begin{equation}
    \begin{aligned}
        \mathrm{Maximize}\quad & Q\bullet Y                                                        \\
        \mathrm{s.t.}  \quad   & Y-xx^T \succeq 0 \text { or } \begin{bmatrix} 1 & x^{T} \\ x & Y \end{bmatrix} \succeq 0 \\
                               & A_i \bullet Y \; (\le, =, \ge) \; b_i, \forall i                  \\
    \end{aligned}
\end{equation}

For MINLP and more specifically for QP, see \cite{belotti_mixed-integer_2013}, \cite{misener_glomiqo_2013}. More recently, spacial branch-and-cut method \cite{chen_spatial_2017} for QP with complex variables.


Branching node and value selection, namely, select which \(x_i, i = 1, ..., N\) and value \(\alpha\), for left and right children, for example:

\[x_i \le \alpha, x_i \ge \alpha\]

\begin{itemize}
    \item \cite{audet_branch_2000} using RLT relaxation and LP, literally, \(W_{ij} \approx x_i x_j , v_i \approx x_i^2\). The branching is essentially based on \(\|W_{ij} - x_ix_j\|, \|v_i - x_i^2\|\), at each node we solve a linear programming relaxation.
    \item \cite{linderoth_simplicial_2005} a B-B based on subdividing the feasible region into the Cartesian product of triangles and rectangles.

\end{itemize}
Branching value:


Some source code to look at:

\begin{itemize}
    \item Couenne: https://www.coin-or.org/Couenne/
    \item
\end{itemize}

\paragraph{Branch for \(v= x^2\)}

Using secant cut \(y = (u + l)x - ul\),

\begin{proposition}
    Branching at \(\alpha = \frac{u + l}{2}\) both minimized the area of both children and the maximum violations.
\end{proposition}
\begin{proof}
    calculate sum of area by simple calculus.
    \begin{equation}
        A(\alpha) = \frac{1}{6}(u^3- l^3) + \frac{1}{2}\left[(u-l)\alpha^2 - (u^2-l^2)\alpha\right]
    \end{equation}
    The minimizer can be calculated. \hfill\qed
\end{proof}


\end{document}