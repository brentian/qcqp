\documentclass[../main]{subfiles}
\title{Primal solution}
\author{Chuwen Zhang}
\date{\today}
\begin{document}
\maketitle
{
    \setcounter{tocdepth}{3}
    \tableofcontents
}
\section{Primal solution}

\subsection{Rank Reduction}
\subsubsection{Randomized}

\begin{equation}
    \hat{X}=(V^\star)^T\left[\sum_{j=1}^d \xi^j(\xi^j)^T\right] V^\star,
    (\xi^j)_i \in N(0, 1), i = 1,\dots,n, j = 1, \dots, d
\end{equation}

This method guarantees the rank of \(d\), while meeting a few challenges,

\begin{itemize}
    \item Cannot guarantee feasibility after randomized for inhomogeneous quadratic constraints
\end{itemize}

\subsection{Homogenize}


\begin{align*}
     & x^TAx + a^Tx  - b \le 0      \\
     & x^TAx + ta^Tx  - bt^2 \le 0  \\
     & \begin{bmatrix} x \\ t \end{bmatrix}^T
    \begin{bmatrix} A & a/2 \\ a^T/2 & -b \end{bmatrix}
    \begin{bmatrix} x \\ t \end{bmatrix} \le 0 \\
\end{align*}

\begin{itemize}
    \item if \(\tilde x=(x, t)\) is a solution, then it can be scaled to \((x/t, 1)\).
    \item we solve by SDP relaxation \(\tilde X \succeq \tilde x\tilde x^T\), let \(\tilde X = VV^T\)
          \begin{align*}
               & A \bullet VV^T = \trace(V^TAV) \le 0 \\
          \end{align*}
    \item find \(\xi\) to make \(A \bullet V\xi_{j}\xi_{j}^TV^T = \trace{(V^TAV\xi_{j}\xi_{j}^T)} \le 0\)
\end{itemize}


\subsection{Second order cone approach}
For the original non-convex inhomogeneous QCQP, in the following form,

\begin{align}
    \mx & x^T Q x+q^T x                                    \\
    \st & x^T Q_i x+q_i^T x \leqslant b_{i}, i= 1,\dots, m \\
        & 0 \leqslant x \leqslant 1
\end{align}

Convexify Q constraints,
\begin{equation}\label{eq:primal.Sigma}
    \begin{aligned}
         & x^T(Q_i+t_i I_n) x+q_i^T x \leqslant b_i+ t_i \cdot s^2,i= 1,\dots, m \\
         & x^T x= s^2
    \end{aligned}
\end{equation}
Let \(\Sigma = \{(x,s)\in \mathbb R^{n+1}\mid (x,s) \textrm{ satisfies \eqref{eq:primal.Sigma}} \}\). Note \(t_i, i=1,\dots, m\) is selected to make \(Q_i+t_i I_n\) p.s.d, which can be achieved easily. \(\Sigma\) is clearly nonconvex, nevertheless, our goal is to find an estimate of \(\|x\|\).

We have,

\begin{itemize}
    \item convex upper bound,
          \begin{equation} \|x\| \le s \end{equation}
    \item unit ball, \(\|u\| = 1\)
          \begin{equation} \|x\| = \max_u u^T x \end{equation}
    \item then the lower bound can be established by piecewise linear functions,
          \begin{equation}
              \|x\| = \max_j u^T_j x \ge s
          \end{equation}
\end{itemize}

We sample \(\{u_j\}_1^N\)
\begin{align}
     & \max_j u^T_j y \ge s                 \\
     & \|x \| \le s                         \\
     & (x, s) \in \Sigma, (y, s) \in \Sigma
\end{align}

which is equivalent to \(N\) independent feasibility problem. We let \(\Sigma_j\) be the piece for \(u_j\)
\end{document}