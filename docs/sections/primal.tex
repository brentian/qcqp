\documentclass[../main]{subfiles}
\title{Primal solution}
\author{Chuwen Zhang}
\date{\today}
\begin{document}
\maketitle
{
    \setcounter{tocdepth}{3}
    \tableofcontents
}
\section{Primal solution}

\subsection{Rank Reduction}
\subsubsection{Randomized}

\begin{equation}
    \hat{X}=(V^\star)^T\left[\sum_{j=1}^d \xi^j(\xi^j)^T\right] V^\star,
    (\xi^j)_i \in N(0, 1), i = 1,\dots,n, j = 1, \dots, d
\end{equation}

This method guarantees the rank of \(d\), while meeting a few challenges,

\begin{itemize}
    \item Cannot guarantee feasibility after randomized for inhomogeneous quadratic constraints
\end{itemize}

\subsection{Homogenize}


\begin{align*}
     & x^TAx + a^Tx  - b \le 0      \\
     & x^TAx + ta^Tx  - bt^2 \le 0  \\
     & \begin{bmatrix} x \\ t \end{bmatrix}^T
    \begin{bmatrix} A & a/2 \\ a^T/2 & -b \end{bmatrix}
    \begin{bmatrix} x \\ t \end{bmatrix} \le 0 \\
\end{align*}

\begin{itemize}
    \item if \(\tilde x=(x, t)\) is a solution, then it can be scaled to \((x/t, 1)\).
    \item we solve by SDP relaxation \(\tilde X \succeq \tilde x\tilde x^T\), let \(\tilde X = VV^T\)
          \begin{align*}
               & A \bullet VV^T = \trace(V^TAV) \le 0 \\
          \end{align*}
    \item find \(\xi\) to make \(A \bullet V\xi_{j}\xi_{j}^TV^T = \trace{(V^TAV\xi_{j}\xi_{j}^T)} \le 0\)
\end{itemize}


\subsection{Second order cone approach}
For the original non-convex inhomogeneous QCQP, in the following form,

\begin{align}
    \mx & x^T Q x+q^T x                                    \\
    \st & x^T Q_i x+q_i^T x \leqslant b_{i}, i= 1,\dots, m \\
        & 0 \leqslant x \leqslant 1
\end{align}

Convexify Q constraints,
\begin{equation}\label{eq:primal.Sigma}
    \begin{aligned}
         & x^T(Q_i+t_i I_n) x+q_i^T x \leqslant b_i+ t_i \cdot s^2,i= 1,\dots, m \\
         & x^T x= s^2
    \end{aligned}
\end{equation}
Let \(\Sigma = \{(x,s)\in \mathbb R^{n+1}\mid (x,s) \textrm{ satisfies \eqref{eq:primal.Sigma}} \}\). Note \(t_i, i=1,\dots, m\) is selected to make \(Q_i+t_i I_n\) p.s.d, which can be achieved easily. \(\Sigma\) is clearly nonconvex, nevertheless, our goal is to find an estimate of \(\|x\|\).

We have,

\begin{itemize}
    \item convex upper bound,
          \begin{equation} \|x\| \le s \end{equation}
    \item unit ball, \(\|u\| = 1\)
          \begin{equation} \|x\| = \max_u u^T x \end{equation}
    \item then the lower bound can be established by piecewise linear functions,
          \begin{equation}
              \|x\| = \max_j u^T_j x \ge s
          \end{equation}
\end{itemize}

We sample \(\{u_j\}_1^N\)
\begin{align}
     & \max_j u^T_j y \ge s                 \\
     & \|x \| \le s                         \\
     & (x, s) \in \Sigma, (y, s) \in \Sigma
\end{align}

which is equivalent to \(N\) independent feasibility problem. We let \(\Sigma_j\) be the piece for \(u_j\)


\subsection{Connection to MSC}
Consider MSC,
\begin{equation}\label{eq.rel.msc}
    \begin{aligned}
        \model{MSC} \quad \mathrm{Maximize: }\quad & y_0 ^T\lambda_0                                     \\
        \mathrm{s.t.} \quad                        & V_i z_i = x                        & i=0,...,m      \\
                                                   & y_i ^T\lambda_i  + a_i^Tx  \le b_i & i=1,...,m      \\
                                                   & y_i \ge z_i \circ z_i              & i=0,...,m      \\
                                                   & y_i^Te \le x^Te                    & i=0, \cdots, m
    \end{aligned}
\end{equation}

Let,
\begin{equation}
    \Omega = \left\{ (x,z,y) :  \begin{aligned}
         & V_i z_i = x                        & i=0,...,m \\
         & y_i ^T\lambda_i  + a_i^Tx  \le b_i & i=1,...,m \\
         & y_i \ge z_i \circ z_i              & i=0,...,m \\
    \end{aligned}\right\}
\end{equation}

Since if \((y^*)^Te = (x^*)^Tx^*\), then \((x^*, y^*)\) is the solution, our problem is related to the above analysis, that is, the optimization problem with the ball constraint.
This motivates the following, let \(\|\xi\|^2 = \|x\|^2 = s\), and we know,
\begin{align}
     & \|x\|^2 = \max_\xi \xi^T x           \\
     & \xi^Tx \ge s - \epsilon, \forall \xi
\end{align}
\begin{align}
    \model{MSC} \quad \mathrm{Maximize: }\quad & y_0 ^T\lambda_0                     \\
    \mathrm{s.t.} \quad                        & (y,z,x) \in \Omega                  \\
                                               & y_i^Te \le t       & i=0, \cdots, m \\
    (\kappa) \quad                             & t= s               & i=0, \cdots, m \\
    (\mu)    \quad                             & \xi^Tx = t         & i=0, \cdots, m \\
                                               & \xi^T\xi \le s     & i=0, \cdots, m
\end{align}

This allows the ALM for bilinear constraint,

\begin{align*}
    \mathscr L\left(x,y,z,\xi,s,\kappa,\mu\right) = - y_0 ^T\lambda_0 + \kappa(t-s) + \mu(\xi^Tx - t) + \frac{\rho}{2}(t-s)^2 + \frac{\rho}{2}(\xi^Tx - s)^2
\end{align*}

The ADMM iteration,

\begin{align*}
    (x,y,z,t)^{k+1} & = {\arg\min}_{(x,y,z)\in\Omega, t\ge 0} L\left(x,y,z,\xi^k,s^k,\kappa^k,\mu^k\right)       \\
    (s, \xi)^{k+1}  & = {\arg\min}_{(s, \xi)\in\mathscr{Q}} L\left((x,y,z,t)^{k+1},\xi,s, \kappa^k, \mu^k\right) \\
    \kappa^{k+1}    & = \kappa^k + \rho\left(t^{k+1}-s^{k+1}\right)                                              \\
    \mu^{k+1}       & = \mu^k + \rho\left( \langle\xi^{k+1}, x^{k+1}\rangle - s^{k+1}\right)                     \\
\end{align*}

\end{document}