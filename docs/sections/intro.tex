\documentclass[../main]{subfiles}

\begin{document}
\section{Introduction}
In this paper, we are concerned with global solutions to nonconvex quadratic constrained quadratic programming (QCQP) problems.


The inhomogeneous QCQP problem can be expressed in the following form:

\begin{equation}
      \label{eq:inhoqcqp}
      \begin{aligned}
            \model{QCQP} ~\mx~ & x^T A_0 x + a_0^T x         \\
            \st ~              & x^T A_i x + a_i^T x \le b_i \\
      \end{aligned}
\end{equation}

The variable is \(x \in \real^n\) and the matrices \(\{A_i, i=0,\ldots, m\}\) are symmetric matrices. We say the problem \eqref{eq:inhoqcqp} is homogeneous if \(a_i = 0, i=0,\ldots, m\). The generic QCQP envelopes a wide range of important applications such as the max-cut problem, quadratic assignment problem, sensor network localization \cite{biswas_semidefinite_2004}, and so forth. Despite the algebraic elegance of QCQP, only a few special cases of the problem are easy in terms of polynomial solvability. It can be shown in many other cases the problem is NP-hard, see \cite[]{pardalos_quadratic_1991, ...}.

Among convex relaxations, semidefinite programming relaxation (SDR) is a widely used technique to solve QCQP approximately. In this approach, the variable \(x\) in the original space is lifted to a rank-one matrix \(X = xx^T\). By placing \(X\) in the space of semidefinite matrices while ignoring the requirement on rank, the SDR produces an upper bound for QCQP that turns out to be very appealing in many applications \cite[]{biswas_semidefinite_2004,biswas_semidefinite_2006}. In general, the solution of SDR (say \(X^*\)) has a rank that is greater than one. A general bound on rank of \(X^*\) is given in \cite[]{pataki, }. It is possible, however, to expect a rank-one solution from SDR if the data matrices come with some special structures, such as the case where the matrices are diagonal\cite[]{burer_exact_2020}, forest structured \cite[]{azuma_exact_2021}, and many others.
% We note \cite[]{wang_tightness_2021}
If one permits only one quadratic constraint, it is also possible to find exact solutions by SDR. see \cite{sturm_cones_2003, ye_new_2003}.

Without a safeguard on the rank of the SDR, a popular approach is to recover \(x\) by sacrificing the quality of objective value by using low-rank approximations. For this vein of research, one refer to \cite[]{so_unified_2008,}. Instead, if a global solution is desired, branch-and-bound type methods can be used to successively solve SDR in smaller subregions \cite[]{audet_branch_2000,vandenbussche_branch-and-cut_2005}. In fact, LP relaxation can be used to replace SDR in nodal problems to achieve cheaper subproblems as well as to take advantage of warm-start techniques. However, the quicker LP relaxations typically produce upper bounds way-off from those by SDR, which greatly increase the number of needed nodes in the searching process.


\end{document}