\documentclass[../main]{subfiles}
\title{Relaxations for QCQP}
\author{Chuwen Zhang}
\date{\today}
\begin{document}
\maketitle
{
  \setcounter{tocdepth}{3}
  \tableofcontents
}

\section{Relaxations for QCQP}

As a convention, we assume data matrices are symmetric, i.e., \(Q, A_i \in \mathcal{S}^n\)

Recall homogeneous QCQP for \(x \in \mathbb R^n\):

\begin{equation}
  \begin{aligned}
    \textrm { Maximize } \quad & x^{T} Q x                                                      \\
    \text { s.t. }  \quad      & x^{T} A_{i} x\; (\le, =, \ge) \; b_{i}, \forall i=1, \ldots, m \\
                               & 0 \le x\le  1
  \end{aligned}
\end{equation}

And inhomogeneous QCQP,

\begin{equation}\label{eq:inhoqcqp}
  \begin{aligned}
    \mathrm{Maximize}\quad & x^TQx +2 q^T x                                   \\
    \mathrm{s.t.}  \quad   & x^{T} A_i x  + 2a_i^Tx   \; (\le, =, \ge) \; b_i \\
                           & 0 \le x\le  1
  \end{aligned}
\end{equation}

We mark some of the trivial techniques below.


One can always reformulate \eqref{eq:inhoqcqp} into a homogeneous problem by increasing the dimension of variables by $1$.

\begin{equation}
  \begin{aligned}
    \textrm { Maximize } \quad & x^{T} Q x  + 2q^Tx        \\
    =                          & \begin{bmatrix}x^T & t\end{bmatrix}
    \begin{bmatrix} Q   & q \\ q^T & o \end{bmatrix}
    \begin{bmatrix} x \\ t\end{bmatrix}                              \\
    \mathrm{s.t.} \quad        & - 1\le t \le 1
  \end{aligned}
\end{equation}

Also, one can solve a symmetric version since \(x^TAx = x^TA^Tx\) by letting,
\begin{equation}
  \tilde A := \frac{A+A^T}{2}
\end{equation}

\subsection{SDP Relaxation}
For \(x \in \mathbb{R}^{n}\), we have: \(x^{T} A_{i} x = A_i \bullet (xx^T)\) and \(xx^T \in \mathcal{S}^n_{+}\), which results in following relaxation using semidefinite cones, also called \textit{lifting} method or Shor relaxation,

\begin{equation}\label{eq:shor_basic}
  \begin{aligned}
    \mathrm{Maximize}\quad & Q\bullet Y   + 2q^T x                                            \\
    \mathrm{s.t.}  \quad   & Y-xx^T \succeq 0 \text { or }\begin{bmatrix} 1 & x^{T} \\ x & Y \end{bmatrix} \succeq 0 \\
                           & A_i \bullet Y +2 a_i^Tx \; (\le, =, \ge) \; b_i, \forall i       \\
                           & 0\le x\le 1
  \end{aligned}
\end{equation}

Notice QCQP with matrix variables can also be reformulated into a SDP based problem, let \(X \in \mathbb{R}^{n\times d}\), then \(X^{T} A_{i} X = A_i \bullet (XX^T)\)
\begin{equation}
  \begin{aligned}
    \mathrm{Maximize}\quad & Q\bullet Y                                                       \\
    \mathrm{s.t.}  \quad   & Y-XX^T \succeq 0 \text { or }\begin{bmatrix} I_d & X^{T} \\ X & Y \end{bmatrix} \succeq 0 \\
                           & A_i \bullet Y \; (\le, =, \ge) \; b_i, \forall i                 \\
  \end{aligned}
\end{equation}

SDP relaxation \eqref{eq:shor_basic} can be unbounded in some case. A simple improvement to \eqref{eq:shor_basic} is to add bounds for the diagonal entries.

\begin{equation}\label{eq:shor}
  \begin{aligned}
    \mathrm{Maximize}\quad & Q\bullet Y   + 2q^T x                                            \\
    \mathrm{s.t.}  \quad   & Y-xx^T \succeq 0 \text { or }\begin{bmatrix} 1 & x^{T} \\ x & Y \end{bmatrix} \succeq 0 \\
                           & A_i \bullet Y +2 a_i^Tx \; (\le, =, \ge) \; b_i, \forall i       \\
                           & 0\le x\le 1                                                      \\
                           & \mathrm{diag}(Y) \le x
  \end{aligned}
\end{equation}

There are many further enhancements to \eqref{eq:shor}, see \cite{bao_semidefinite_2011} for discussion on the strength of different relaxations. Here we discuss a few widely used methods using copositive cones and reformulation-linearization-techniques (RLT) cuts.


Consider the dual of primal SDP relaxation.
\begin{equation}
  \begin{aligned}
    L = & \max_{x, X} \; Q\bullet X + q^Tx + \sum_i \left( \lambda_i A_i \bullet X  + \lambda_i a_i^Tx - \lambda_i b_i \right ) + \mu^Tx - \mu^Te \\
        & - \begin{bmatrix}X & x \\ x^T & 1\end{bmatrix} \bullet \begin{bmatrix}Y & y \\ y^T & 1\end{bmatrix}                                                                         \\
    =   & \max_{x, X} \; - \lambda^Tb - \mu^Te - 1+X\bullet (Q + \sum_i\lambda_iA_i - Y) + x^T(q + \sum_i \lambda_i a_i + \mu - 2 y)
  \end{aligned}
\end{equation}

And thus the dual,

\begin{equation}
  \begin{aligned}
    \min_{\lambda, \mu, y, Y}\quad & \lambda^Tb + \mu^Te + 1                    \\
    \mathrm{s.t.} \quad            & Y = Q + \sum_i \lambda_i A_i               \\
                                   & q + \sum_i \lambda_i a_i + \mu - 2 y \le 0 \\
                                   & Y \succeq yy^T                             \\
                                   & \lambda, \mu \ge 0
  \end{aligned}
\end{equation}
\subsection{SOCP Relaxation}

We now consider another way of relaxing the original QCQP problem. Consider symmetric indefinite matrix \(Q \in \mathcal{S}^n\) and its spectral decomposition.

\begin{equation}
  \begin{aligned}
     & Q = V\Lambda V ^T =\sum_j^n \lambda_j v_j  v_j^T \\
     & \Lambda = \mathrm{diag}(\lambda)
  \end{aligned}
\end{equation}

Without loss of generality, we assume first \(r\) eigenvalues are positive, \(\lambda_1, ..., \lambda_r \ge 0, r\le n \). The quadratic form \(x^TQx\) can also be partitioned into positive and negative parts:

\begin{equation}\label{eq:qpform}
  \begin{aligned}
    x^TQx = \sum_{j=1}^r \lambda_j x^T v_j v_j^T x + \sum_{j=r+1}^n \lambda_j x^T v_j v_j^T x
  \end{aligned}
\end{equation}

By letting \(y_j \ge z_j^2, z_j = v_j^T x, j = 1, \cdots, n\), we introduce \(n\) (small) quadratic cones, then \eqref{eq:qpform} can be rewritten as:

\begin{equation}
  \begin{aligned}
    x^TQx \le \lambda^Ty,\; (y_j, v^T_j x) \in \mathcal{Q}^2
  \end{aligned}
\end{equation}

A natural Many-Small-Cone (MSC) relaxation to  QCQP can be written as:










\end{document}