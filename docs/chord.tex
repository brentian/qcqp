%%
% Copyright (c) 2017 - 2020, Pascal Wagler;
% Copyright (c) 2014 - 2020, John MacFarlane
%
% All rights reserved.
%
% Redistribution and use in source and binary forms, with or without
% modification, are permitted provided that the following conditions
% are met:
%
% - Redistributions of source code must retain the above copyright
% notice, this list of conditions and the following disclaimer.
%
% - Redistributions in binary form must reproduce the above copyright
% notice, this list of conditions and the following disclaimer in the
% documentation and/or other materials provided with the distribution.
%
% - Neither the name of John MacFarlane nor the names of other
% contributors may be used to endorse or promote products derived
% from this software without specific prior written permission.
%
% THIS SOFTWARE IS PROVIDED BY THE COPYRIGHT HOLDERS AND CONTRIBUTORS
% "AS IS" AND ANY EXPRESS OR IMPLIED WARRANTIES, INCLUDING, BUT NOT
% LIMITED TO, THE IMPLIED WARRANTIES OF MERCHANTABILITY AND FITNESS
% FOR A PARTICULAR PURPOSE ARE DISCLAIMED. IN NO EVENT SHALL THE
% COPYRIGHT OWNER OR CONTRIBUTORS BE LIABLE FOR ANY DIRECT, INDIRECT,
% INCIDENTAL, SPECIAL, EXEMPLARY, OR CONSEQUENTIAL DAMAGES (INCLUDING,
% BUT NOT LIMITED TO, PROCUREMENT OF SUBSTITUTE GOODS OR SERVICES;
% LOSS OF USE, DATA, OR PROFITS; OR BUSINESS INTERRUPTION) HOWEVER
% CAUSED AND ON ANY THEORY OF LIABILITY, WHETHER IN CONTRACT, STRICT
% LIABILITY, OR TORT (INCLUDING NEGLIGENCE OR OTHERWISE) ARISING IN
% ANY WAY OUT OF THE USE OF THIS SOFTWARE, EVEN IF ADVISED OF THE
% POSSIBILITY OF SUCH DAMAGE.
%%


% @modified: Chuwen <chuwzhang@gmail.com>
% Options for packages loaded elsewhere
%
\PassOptionsToPackage{unicode}{hyperref}
\PassOptionsToPackage{hyphens}{url}
\PassOptionsToPackage{dvipsnames,svgnames*,x11names*,table}{xcolor}
\documentclass[
  a4paper,
  ,tablecaptionabove
]{scrartcl}
\usepackage{lmodern}
\usepackage{setspace}
\setstretch{1.2}
\usepackage{amssymb,amsmath}
\numberwithin{equation}{section}
\usepackage{ifxetex,ifluatex}
\usepackage{unicode-math}
\defaultfontfeatures{Scale=MatchLowercase}
\defaultfontfeatures[\rmfamily]{Ligatures=TeX,Scale=1}

% Use upquote if available, for straight quotes in verbatim environments
\IfFileExists{upquote.sty}{\usepackage{upquote}}{}
\IfFileExists{microtype.sty}{% use microtype if available
  \usepackage[]{microtype}
  \UseMicrotypeSet[protrusion]{basicmath} % disable protrusion for tt fonts
}{}
\makeatletter
\@ifundefined{KOMAClassName}{% if non-KOMA class
  \IfFileExists{parskip.sty}{%
    \usepackage{parskip}
  }{% else
    \setlength{\parindent}{0pt}
    \setlength{\parskip}{6pt plus 2pt minus 1pt}}
}{% if KOMA class
  \KOMAoptions{parskip=half}}
\makeatother
\usepackage{xcolor}
\definecolor{default-linkcolor}{HTML}{A50000}
\definecolor{default-filecolor}{HTML}{A50000}
\definecolor{default-citecolor}{HTML}{4077C0}
\definecolor{default-urlcolor}{HTML}{4077C0}
\IfFileExists{xurl.sty}{\usepackage{xurl}}{} % add URL line breaks if available
\IfFileExists{bookmark.sty}{\usepackage{bookmark}}{\usepackage{hyperref}}
\hypersetup{
  pdfauthor={Chuwen},
  hidelinks,
  breaklinks=true,
  pdfcreator={LaTeX via pandoc with the Eisvogel template}}
\urlstyle{same} % disable monospaced font for URLs
\usepackage[margin=2.5cm,includehead=true,includefoot=true,centering,]{geometry}
% add backlinks to footnote references, cf. https://tex.stackexchange.com/questions/302266/make-footnote-clickable-both-ways
\usepackage{footnotebackref}
\usepackage{graphicx,grffile}
\makeatletter
\def\maxwidth{\ifdim\Gin@nat@width>\linewidth\linewidth\else\Gin@nat@width\fi}
\def\maxheight{\ifdim\Gin@nat@height>\textheight\textheight\else\Gin@nat@height\fi}
\makeatother
% Scale images if necessary, so that they will not overflow the page
% margins by default, and it is still possible to overwrite the defaults
% using explicit options in \includegraphics[width, height, ...]{}
\setkeys{Gin}{width=\maxwidth,height=\maxheight,keepaspectratio}
\setlength{\emergencystretch}{3em}  % prevent overfull lines
\providecommand{\tightlist}{%
  \setlength{\itemsep}{0pt}\setlength{\parskip}{0pt}}
\setcounter{secnumdepth}{3}

% Make use of float-package and set default placement for figures to H.
% The option H means 'PUT IT HERE' (as  opposed to the standard h option which means 'You may put it here if you like').
\usepackage{float}
\floatplacement{figure}{H}
\usepackage{amsthm}
\usepackage[ruled,vlined]{algorithm2e}
\newtheorem{theorem}{Theorem}[section]
\newtheorem{corollary}{Corollary}[theorem]
\newtheorem{lemma}[theorem]{Lemma}
\newtheorem{prop}{Proposition}

%
% language specification
%
% If no language is specified, use English as the default main document language.
%
\usepackage{polyglossia}
\setmainlanguage[]{english}

\usepackage[UTF8, heading=true]{ctex}
\usepackage{booktabs}
\definecolor{tufeijilk}{RGB}{68,87,151}
\hypersetup{colorlinks=true,linkcolor=tufeijilk,urlcolor=cyan,citecolor=tufeijilk}
\newlength{\cslhangindent}
\setlength{\cslhangindent}{1.5em}
\newenvironment{cslreferences}%
{}%
{\par}

%
% break urls
%
\PassOptionsToPackage{hyphens}{url}

%
% When using babel or polyglossia with biblatex, loading csquotes is recommended
% to ensure that quoted texts are typeset according to the rules of your main language.
%
\usepackage{csquotes}

%
% captions
%
\definecolor{caption-color}{HTML}{777777}
\usepackage[font={stretch=1.2}, textfont={color=caption-color}, position=top, skip=4mm, labelfont=bf, singlelinecheck=false, justification=raggedright]{caption}
\setcapindent{0em}

%
% blockquote
%
\definecolor{blockquote-border}{RGB}{221,221,221}
\definecolor{blockquote-text}{RGB}{119,119,119}
\usepackage{mdframed}
\newmdenv[rightline=false,bottomline=false,topline=false,linewidth=3pt,linecolor=blockquote-border,skipabove=\parskip]{customblockquote}
\renewenvironment{quote}{\begin{customblockquote}\list{}{\rightmargin=0em\leftmargin=0em}%
    \item\relax\color{blockquote-text}\ignorespaces}{\unskip\unskip\endlist\end{customblockquote}}

%
% heading color
%
\definecolor{heading-color}{RGB}{40,40,40}
\addtokomafont{section}{\color{heading-color}}
% When using the classes report, scrreprt, book,
% scrbook or memoir, uncomment the following line.
%\addtokomafont{chapter}{\color{heading-color}}

%
% variables for title and author
%
\usepackage{titling}

%
% tables
%
\usepackage{array}
\usepackage{multirow}
\usepackage{longtable}
%
% remove paragraph indention
%
\setlength{\parindent}{0pt}
\setlength{\parskip}{6pt plus 2pt minus 1pt}
\setlength{\emergencystretch}{3em}  % prevent overfull lines

%
%
% Listings
%
%


%
% header and footer
%
\usepackage{fancyhdr}

\fancypagestyle{eisvogel-header-footer}{
  \fancyhead{}
  \fancyfoot{}
  \lhead[\today]{}
  \chead[]{}
  \rhead[]{\today}
  \lfoot[\thepage]{Chuwen}
  \cfoot[]{}
  \rfoot[Chuwen]{\thepage}
  \renewcommand{\headrulewidth}{0.4pt}
  \renewcommand{\footrulewidth}{0.4pt}
}
\pagestyle{eisvogel-header-footer}


\usepackage{natbib}
\bibpunct[, ]{(}{)}{,}{a}{}{,}%
\def\bibfont{\small}%
\def\bibsep{\smallskipamount}%
\def\bibhang{24pt}%
\def\newblock{\ }%
\def\BIBand{and}%
\title{Global QCQP Solver}
\author{Chuwen Zhang}
\date{\today}


\usepackage{subfiles}
\usepackage{graphicx} 
\usepackage{subfiles} 
\usepackage{subfig}
\usepackage[table]{xcolor}
\usepackage{mathrsfs}
\usepackage[font=small,labelfont=bf]{caption}
\begin{document}
% \subfile{sections/intro.tex}
% \subfile{sections/sdp.tex}
% \subfile{sections/app.tex}
% \subfile{sections/app.qkp.tex}
% \subfile{sections/bc.tex}
% \subfile{sections/chord.tex}


\section{Graph Theory Review}

\subsection{Miscellanea}
\begin{itemize}
      \item Complement of a graph is the graph with same vertices and complement of edges.
            \[G^\star\left(V, V^2 \backslash E\right)\]

\end{itemize}
\subsection{Packing, Covering, et cetera}



\subsubsection*{Packing}
Set packing: finite set \(S\) and a list of subsets of \(S\). Then, the set packing problem asks if some \(k\) subsets in the list are pairwise disjoint, and its optimization problem determining the maximum \(k\)
\begin{itemize}
      \item \textbf{IS} (independent) stable set: set \(S \subseteq V\) is a stable of an
            undirected graph \(G = (V, E)\) if no two vertices in \(S\) are
            adjacent, i.e., the induced edge set is empty.
      \item Sometimes referred to as ``vertex packing''
      \item The size of the
            largest stable set is called the stable set number of the graph,
            denoted \(\alpha(G)\). The problem \textbf{(MIS) Maximum independent set} denotes the problem of finding the largest such set.
            \begin{equation}
                  \begin{aligned}
                         & E(S) \equiv \{(v, w) \in E | v, w \in S\} \\
                         & \mathscr S = \{S: E(S)= \emptyset\}       \\
                         & \alpha (G) = \max_{S\in \mathscr S} |S|
                  \end{aligned}
            \end{equation}

      \item Vertex coloring: assignment of vertices in a way that no adjacent vertices are of same color, corresponds to a partition of its vertex set into independent subsets. the \textbf{chromatic number} is defined as the minimum needed colors, has the following relation,
            \begin{equation}
                  \chi(G) \ge \frac{|G|}{\alpha(G)}
            \end{equation}
\end{itemize}

\subsubsection*{Covering}
Set covering, formally, given a universe \(\) and a collection of subsets \(\mathscr S = \{S\}\), the set cover problem is to find such cover, i.e., \(\mathscr U \subseteq \bigcup_{S\in\mathscr S} S \)
\begin{itemize}
      \item Vertex cover: a subset \(V' \subseteq V\) such that \(V'\) covers at least one endpoint of edges. It is not strictly a cover since it is not set of sets.
\end{itemize}

\subsubsection*{Clique and Completeness}
\begin{itemize}
      \tightlist
      \item[\(\checkmark\)] fun fact: in the social sciences, clique is a group of individuals who interact with one another and share similar interests
      \item Complete (induced subgraph): a (sub)-graph is complete if all vertices are pairwise adjacent.
      \item (Maximal) Clique: a clique is a set of vertices \(C \subseteq V\) that induces a (maximal) complete subgraph. Normally we use two terms interchangeably except that maximal is declared.
      \item \textbf{Maximal}: i.e., it cannot be extended by including one more adjacent vertex
      \item Simplicial: A vertex \(v\) is simplicial if its neighborhood \(adj(v)\) is complete.
      \item Clique cover (CC): a clique cover \(\mathscr C\) of \(G\) is a set of
            cliques that cover the vertex set \(V\). The clique cover number
            \(\bar \chi (G)\) is the number of cliques in the cover. \textbf{(MCC) Minimum clique cover} is the CC with minimized number of cliques.

            \begin{equation}
                  \begin{aligned}
                         & V \subseteq \mathscr C \equiv \bigcup_k C_k    \\
                         & \bar \chi (G) = \min_{\mathscr C} |\mathscr C|
                  \end{aligned}
            \end{equation}
      \item Clique edge cover (CEC) Intersection graph respect a cover \(\{S_i\}\), is a graph by representative vertex \(v_i, \forall i\) as vertex set \(V\), and edge set,

            \[E(G)=\left\{(v_{i}, v_{j}) \mid i \neq j, S_{i} \cap S_{j} \neq \emptyset \right\}\]

            intersection number is the smallest number of elements of the cover: \(|\bigcup_i S_i|\)
\end{itemize}



\subsubsection*{Remark: complementarity}
We mark the relationship and paraphrases here
\begin{itemize}
      \item A set \(S\) in \(G\) is independent \textbf{iff.} \(V\backslash S\) is a vertex cover
      \item A set in \(G\) is independent \textbf{iff.} it is a clique in the \(G^\star\)
      \item For a clique cover \(\mathscr C\) and independent set \(S\), we always have, \(|\mathscr C| \ge |S|\) and,
            \[\bar \chi (G) \ge \alpha(G)\]
            \textbf{PF}: \(\forall k, C_k \) covers at most one vertex in \(S\). The clique cover number and independent number is obvious by taking min and max on left and right, respectively.
\end{itemize}

\subsubsection*{Remark: Complexity}

Generally, finding a MCC is NP-hard, and its decision version is NP-complete. Also, MIS is a strongly NP-hard problem.

If \(G\) is chordal, we have the following theorem.

\begin{theorem}
      (\cite{gavril_algorithms_1972}, \cite{vandenberghe_chordal_2015})
      \(\bar \chi (G) = \alpha(G)\) if \(G\) is chordal,
      polynomial time algorithms exist for both numbers, see \cite{gavril_algorithms_1972}
\end{theorem}


\subsection{Chordal}

chord and chordal graph: an undirected graph \(G\) is called chordal
(or \textbf{triangulated}, or a rigid circuit) if every cycle of
length greater than or equal to 4 has at least one chord.

\begin{itemize}
      \tightlist
      \item non-chordal graphs can always be chordal extended, i.e., extended to  a chordal graph, by adding additional edges to the original graph
\end{itemize}

ordering: an ordering \(\sigma \{1, 2, ... , n\} \Rightarrow V\) can
also be interpreted as a sequence of vertices
\(\sigma = (\sigma(1), ... , \sigma(n))\). We refer to
\(\sigma^{−1} (v)\) as the index of vertex \(v\) of such ordering.

\begin{itemize}
      \tightlist
      \item
            denote a ordered graph as \(G_\sigma\)
\end{itemize}
\bibliography{headers/qcqp}
\bibliographystyle{headers/spmpsci}
\addcontentsline{toc}{section}{References}

% finish off


\section*{Appendix}\label{sec:appendix}

\end{document}