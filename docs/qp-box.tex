\documentclass[a4paper, 10pt]{article}
\usepackage[english]{babel}
\usepackage{amsmath, amsthm, subfiles, bm, hyperref, graphicx}
\usepackage{caption, longtable, booktabs}
\usepackage{cancel}
\usepackage[top=2cm, bottom=4.5cm, left=3cm, right=3.5cm]{geometry}
% my cmd
%%%%%%%%%%%%%%%%
% start my commands
%%%%%%%%%%%%%%%%
\newcommand{\lm}{\lambda_\textrm{max}}
\newcommand{\trace}{\mathbf{trace}}
\newcommand{\diag}{\mathbf{diag}}
\newcommand{\model}[1]{(\textbf{#1})}
\newcommand{\mx}{\mathbf{\max}\;}
\newcommand{\mn}{\mathbf{\min}\;}
\newcommand{\st}{\mathrm{s.t.\;}}
\newcommand{\ex}{\mathbf E}
\newcommand{\dx}{\;\bm dx}
\newcommand{\pr}{\mathbf P}
\newcommand{\id}{\mathbf I}
\newcommand{\bp}{\mathbb P}
\newcommand{\be}{\mathbb E}
\newcommand{\bi}{\mathbb I}
\newcommand{\bxi}{{\bm \xi}}
\newcommand{\va}{\mathbf{Var}}
\newcommand{\dif}{\mathbf{d}}
\newcommand{\minp}[2]{\min\{#1, #2\}}
\newcommand{\intp}{\mathbf{int}}
\newcommand{\apex}{\mathbf{apex}}
\newcommand{\conv}{\mathbf{conv}}
\newcommand{\red}[1]{\textcolor{red}{#1}}
\title{QCQP: The Low-Rank Case}
\author{Chuwen Zhang}
\date{\today}


\usepackage{subfiles}
\usepackage{graphicx} 
\usepackage{subfiles} 
\usepackage{subfig}
\usepackage[table]{xcolor}
\usepackage[font=small,labelfont=bf]{caption}



\begin{document}
\maketitle
% {
%     \setcounter{tocdepth}{3}
%     \tableofcontents
% }
\section{The Low-Rank QCQP}

Consider QCQP,
\begin{equation}
    \label{eq:inhoqcqp}
    \begin{aligned}
        \model{QCQP} \quad \mx \quad & x^T Qx + q^T x                           \\
        \textrm{s.t.} \quad          & x^T A_i x + a_i^T x \le b_i, i=1, ..., m \\
    \end{aligned}
\end{equation}
In addition, we may have \(\|x\| \le \delta\) or box \(x \in [0, 1]^n\).

We further assume each matrix \(A\) in \(Q, A_i, i=1,...,m\) has the spectral decomposition with eigenvalues \(\{\lambda_j\}_{j=1}^n\). We further assume \(\lambda_1 \le \lambda_2 \le ... \le \lambda_n\) and that first \(r\) eigenvalues are negative.

For clarity, we say \(A_i\) is \emph{rank-\(r\) indefinite} if first \(r\) eigenvalues are negative. In comparison, \(Q\) is rank-\(r\) indefinite if last \(r\) eigenvalues are positive accounting for a maximization problem.

\section{Rank-\(1\) Indefinite Matrix}

Consider the case where \(r=1\).
\subsection{Homogeneous quadratic inequality}
Consider a homogeneous quadratic inequality with matrix \(Q\),
\begin{equation}
    \Omega = \{x: x^TQx \le 0\}
\end{equation}


Note if \(r = 0\), \( \Omega\) is readily a convex set. If there is only one negative \(\lambda_j\), then it is \emph{second-order cone representable} (SOCr):

\begin{equation}
    \Omega = \{x: \|R^Tx\| \le \sqrt {-\lambda_j} v_j^Tx \} \cup \{x: \|R^Tx\| \le - \sqrt {-\lambda_j} v_j^Tx \}
\end{equation}

Then we can decompose the problem into two disjunctions, each of which is convex.
\subsection{\(Q\) is rank-\(1\) indefinite}

If \(m = 0\), that is, there is no further constraints, then \(x^TQx = x^T(\lambda\cdot vv^T-RR^T)x + q^Tx\), \eqref{eq:inhoqcqp} can be rewritten into the following form,

\begin{equation}\label{eq:qp_unc_r1}
    \begin{aligned}
        \mx \quad & z                                              \\
        \st \quad & z + x^TRR^Tx - q^Tx \le \lambda \cdot (v^Tx)^2
    \end{aligned}
\end{equation}

then,
\begin{equation}\label{eq:qp_unc_r1_conic}
    \begin{aligned}
        \mx \quad & z                                                         \\
        \st \quad & \left\|R^Tx \right \|^2 \le \lambda \cdot \rho + q^Tx - z \\
                  & (v^Tx)^2 \le \rho                                         \\
    \end{aligned}
\end{equation}
Obviously, \eqref{eq:qp_unc_r1_conic} is a convex relaxation of \eqref{eq:qp_unc_r1}. The problem is unbounded without regularization on \((\rho, v^Tx)\) that can be constructed from the boundary of \(x\). In practice, such boundary could be a box or a ball by \(2\)-norm.

\subsection{Inhomogeneous quadratic inequality}
\section{Q with positive \(r\) eigenvalues}
Now we consider QCQP wi
\section{Review}
\end{document}
