\PassOptionsToClass{a4paper,10pt}{article} % for article mode 
\PassOptionsToClass{utf8,aspectratio=1610,10pt}{beamer} % for beamer, handout, trans modes
\newcommand*{\BeamerswitchSpawn}[1]{
    \ShellEscape{... -jobname=\jobname#1 \jobname} %
}
\documentclass{beamerswitch}
\mode<article>{
  \usepackage[top=3cm, bottom=3cm, left=3cm, right=2cm]{geometry}
}
\mode<presentation>{
  
}
\usepackage{amsmath, subfiles, bm, hyperref, graphicx}
\usepackage{mathrsfs}
\usepackage{caption, longtable, booktabs}
\usepackage{cancel}

% my cmd
%%%%%%%%%%%%%%%%
% start my commands
%%%%%%%%%%%%%%%%
\newcommand{\lm}{\lambda_\textbf{max}}
\newcommand{\trace}{\textbf{trace}}
\newcommand{\diag}{\textbf{diag}}
\newcommand{\rank}{\textbf{rank}}
\newcommand{\model}[1]{(\textbf{#1})}
\newcommand{\mx}{\mathbf{\max}\;}
\newcommand{\mn}{\mathbf{\min}\;}
\newcommand{\st}{\mathrm{s.t.\;}}
\newcommand{\ex}{\mathbf E}
\newcommand{\dx}{\;\bm dx}
\newcommand{\pr}{\mathbf P}
\newcommand{\id}{\mathbf I}
\newcommand{\bp}{\mathbb P}
\newcommand{\be}{\mathbb E}
\newcommand{\bi}{\mathbb I}
\newcommand{\bxi}{{\bm \xi}}
\newcommand{\va}{\mathbf{Var}}
\newcommand{\dif}{\mathbf{d}}
\newcommand{\minp}[2]{\min\{#1, #2\}}
\newcommand{\intp}{\mathbf{int}}
\newcommand{\apex}{\mathbf{apex}}
\newcommand{\conv}{\mathbf{conv}}
\newcommand{\red}[1]{\textcolor{red}{#1}}
\newcommand{\redsf}[1]{\textcolor{red}{\textsf{#1}}}
\newcommand{\real}{\mathbb{R}}

%%%%%%%%%%%%%%%%
% finish my commands
%%%%%%%%%%%%%%%%
\usefonttheme[onlymath]{serif}
\setbeamertemplate{theorems}[numbered]
\setbeamertemplate{section in toc}[sections numbered]
\setbeamertemplate{subsection in toc}[subsections numbered]
\setbeamertemplate{caption}[numbered]
% default navigation
\defbeamertemplate{navigation symbols}{mydefault}{%
    \usebeamerfont{footline}%
    \usebeamercolor[fg]{footline}%
    \hspace{1em}%
    \insertframenumber/\inserttotalframenumber
}
% standout navigation
\defbeamertemplate{navigation symbols}{standout}{%
  \setbeamertemplate{navigation symbols}{}%
}

\BeforeBeginEnvironment{frame}{%
  \setbeamertemplate{navigation symbols}[mydefault]%
}

\makeatletter
\define@key{beamerframe}{standout}[true]{%
  \setbeamertemplate{navigation symbols}[standout]
}
\makeatother

\title{QCQP: Progress Report}
\begin{document}

\author{
  Chuwen Zhang
}
\frame{\titlepage}

\begin{frame}[allowframebreaks]{Convex QP}
  Recall \cite{ye_extension_1989}, \cite{ye_affine_1992}
  \begin{align*}
    \min q(x)\doteq ~ & x^TQx + q^Tx    \\
    \st             ~ & Ax = b, x \ge 0
  \end{align*}
  We solve by successive trust region subproblems with affine scaling
  \begin{align*}
    \min q(x)\doteq ~ & x^TQx + q^Tx                               \\
    \st             ~ & Ax = b, x \ge 0                            \\
                      & \|(D^k)^{-1}(x - x^k)\|^2  \le \beta^2 < 1
  \end{align*}

  \begin{enumerate}
    \item  Affine scaling: \(D^k = \diag(x^k)\)
    \item  or add a barrier term: \(- \ln(x)^Te \approx - \left(\mu^{k+1} \alpha / \sqrt{n}\right) e^{T} D^{-1}\left(x-x^{k}\right)\)
    \item  Solve QPS at iteration \(k\) by solving a linear equation
  \end{enumerate}
\end{frame}
\begin{frame}[allowframebreaks]{Adopt interior trust region for diagonalized QP}
  Consider box QP in (diagonalized) standard form, \(x \in [0, u]\), we have the the quadratic subproblem (QPS) using the interior trust region method,
  \begin{equation}\label{eq.interior.boxqp}
    \begin{aligned}
      \max ~ & \lambda^T y + q^Tx  + \mu \sum_j \ln x_j & \approx \lambda^T y + q^Tx + \frac{\alpha \mu}{\sqrt n} e^T(Dx-e) \\
      \st ~  & x_j^2 = y_j                              & (\gamma, \Gamma = \diag(\gamma))                                  \\
             & x_j \le u_j                              & (\theta)                                                          \\
             & \|Dx - e\| \le \beta^2                   & (\kappa)
    \end{aligned}
  \end{equation}
  where \(D = \diag(x^k)^{-1}\) is the affine scaling matrix.
  \framebreak

  Lagrangian function,
  \begin{align*}
    L ~ & = -\lambda^Ty - q^Tx - \frac{\alpha \mu}{\sqrt n} e^T(Dx-e) \\
        & + x^T\Gamma x - \gamma^Ty + \theta^Tx - \theta^Tu           \\
        & + \kappa x^TD^2x - 2\kappa e^TDx + \kappa(e^Te - \beta^2)
  \end{align*}
  The first order condition for QPS,

  \begin{align*}
    \xi                                    =  \frac{\alpha \mu}{\sqrt n},~ -\Lambda - \Gamma & = 0                          \\
    2(\Gamma + \kappa \cdot D^2)x + \theta                                                   & = q + De\cdot(\xi + 2\kappa) \\
    \theta \circ (x - u)                                                                     & = 0                          \\
    \kappa \left (\|Dx - e\|^2 - \beta^2   \right )                                          & = 0
  \end{align*}
  Similar to \cite{ye_extension_1989}, it is possible to drop complementary for ellipsoid condition and \(\kappa\)

  \framebreak
  Add second order condition,
  \begin{align*}
    S(\kappa)                             & = \kappa \cdot D^2 - \red{\Lambda} \succeq 0 & \dagger  \\
    2S (\kappa) \cdot \tilde \theta       & = \theta                                     &          \\
    2S (\kappa) \cdot (x + \tilde \theta) & = q + De\cdot(\xi + 2\kappa)                 & \ddagger \\
  \end{align*}

  \begin{enumerate}
    \item We can clearly construct a lower bound for \(\kappa\), by making \(\dagger\) psd and RHS of \(\ddagger\) nonnegative.
    \item We know \(x + \tilde \theta \ge 0, x\le u \), by suitably choose \(\kappa > 0\), we can always find \(x + \tilde \theta \ge 0\)
    \item Next we find \(\tilde \theta \) to make \(x \le u, \theta \circ (x - u) = 0\)
  \end{enumerate}
\end{frame}


\begin{frame}[allowframebreaks]{Bibliography}
  \bibliography{headers/qcqp}
  \bibliographystyle{apalike}
\end{frame}
\end{document}
