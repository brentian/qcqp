\documentclass[aspectratio=1610, 10pt]{beamer}
\usepackage[english]{babel}
\usepackage{amsmath, amsthm, subfiles, bm, hyperref, graphicx}
% my cmd
%%%%%%%%%%%%%%%%
% start my commands
%%%%%%%%%%%%%%%%
\newcommand{\lm}{\lambda_\textrm{max}}
\newcommand{\trace}{\mathbf{trace}}
\newcommand{\diag}{\mathbf{diag}}
\newcommand{\model}[1]{(\textbf{#1})}
\newcommand{\mx}{\mathbf{\max}\;}
\newcommand{\mn}{\mathbf{\min}\;}
\newcommand{\st}{\mathrm{s.t.\;}}
\newcommand{\ex}{\mathbf E}
\newcommand{\dx}{\;\bm dx}
\newcommand{\pr}{\mathbf P}
\newcommand{\id}{\mathbf I}
\newcommand{\bp}{\mathbb P}
\newcommand{\be}{\mathbb E}
\newcommand{\bi}{\mathbb I}
\newcommand{\bxi}{{\bm \xi}}
\newcommand{\va}{\mathbf{Var}}
\newcommand{\dif}{\mathbf{d}}
\newcommand{\minp}[2]{\min\{#1, #2\}}
\newcommand{\intp}{\mathbf{int}}
\newcommand{\apex}{\mathbf{apex}}
\newcommand{\conv}{\mathbf{conv}}
\newcommand{\red}[1]{\textcolor{red}{#1}}
\usefonttheme[onlymath]{serif}

%%%%%%%%%%%%%%%%
% finish my commands
%%%%%%%%%%%%%%%%

% theorem environments
\setbeamertemplate{footline}[frame number]


\makeatletter
\patchcmd{\beamer@sectionintoc}
{\ifnum\beamer@tempcount>0}
{\ifnum\beamer@tempcount>-1}
{}
{}
\beamer@tocsectionnumber=0
\makeatother

% set the toc to be using the number
\setbeamertemplate{section in toc}[sections numbered]
\setbeamertemplate{subsection in toc}[subsections numbered]

\hypersetup{pdfstartview={Fit}}

\AtBeginSection[]
{
    \begin{frame}
        \frametitle{Table of Contents}
        \tableofcontents[        currentsection,         currentsubsection,         subsectionstyle=show/shaded/hide,]
    \end{frame}
}


\defbeamertemplate{section page}{mine}[1][]{%
    \begin{centering}
        \vskip1em\par
        \begin{beamercolorbox}[sep=12pt, center]{part title}
            \usebeamerfont{section title}\insertsection\par
        \end{beamercolorbox}
    \end{centering}
}

\defbeamertemplate{subsection page}{mine}[1][]
{
    \begingroup
    \begin{beamercolorbox}[sep=8pt, center]{subsection title}
        \usebeamerfont{subsection title}\insertsubsection\par
    \end{beamercolorbox}
    \endgroup
}


\setbeamertemplate{section page}[mine]
\setbeamertemplate{subsection page}[mine]

\begin{document}


\title{SOCP + SDP Relaxations for QCQP}

\author{
  Chuwen Zhang
}


\maketitle
\begin{frame}{The QCQP}
  We consider the QCQP,
  \begin{equation}
    \label{eq:inhoqcqp}
    \begin{aligned}
      \model{QCQP} \quad \mx \quad & x^T Qx + q^T x                                    \\
      \textrm{s.t.} \quad          & x^T A_i x + a_i^T x \hspace{0.27em} (\le, =, \ge)
      \hspace{0.27em} b_i                                                              \\
                                   & 0 \le x \le 1
    \end{aligned}
  \end{equation}
  For simplicity we first consider \(m = 0\), i.e.,
  \begin{equation}
    \label{eq:inhoqcqp.box}
    \begin{aligned}
      \model{QCQP} \quad \mx \quad & x^T Qx + q^T x \\
      \textrm{s.t.} \quad          & 0 \le x \le 1
    \end{aligned}
  \end{equation}
\end{frame}
\begin{frame}[allowframebreaks]{Some Existing Schemes to Mix SOCP or Small SDP Cones}
  Now assume we use second-order cones. No lifted matrix \(X \succeq 0\) is allowed.

  \textbf{Method 1},

  \textbf{Method 2},

\end{frame}

\begin{frame}[allowframebreaks]{A Simple Mixed Cone Decomposition}
  We partition \(Q\) into a \(n\times n\) matrix, \(Q_0\) plus a set of small matrices \(\mathcal C = \{Q_c\}_{c=1}\)
  Let \(Q = Q_0 \bigoplus \sum_c \bigoplus Q_c\)

  \begin{enumerate}
    \item for example, we have a large \(Q_0\) and a small \(2\times 2\) matrix \(Q_1\),
          \begin{equation*}
            x^T Q x = x^T \begin{bmatrix}
              q_{11}  &        & q_{1 j} &        & q_{1 i} &        &         \\
                      & \ddots &         &        &         &        &         \\
              q_{j 1} &        & 0       &        & 0       &        &         \\
                      &        &         & \ddots &         &        &         \\
              q_{i 1} &        & 0       &        & 0       &        &         \\
                      &        &         &        &         & \ddots &         \\
                      &        &         &        &         &        & q_{n n}
            \end{bmatrix} x + Q_c \bullet X_c
          \end{equation*}

          \framebreak
    \item similarly, we can construct a set of small cones
          \begin{align*}
            x^TQx & = x^TQ_0x + \sum_c Q_c \bullet X_c \\
            X_c   & \succeq x_c x_c^T
          \end{align*}
  \end{enumerate}
\end{frame}

\begin{frame}[allowframebreaks]{Questions \& Future Work}
  \begin{enumerate}
    \item todo
  \end{enumerate}


\end{frame}
\begin{frame}[allowframebreaks]{Bibliography}
  \bibliography{headers/qcqp}
  \bibliographystyle{alpha}
\end{frame}
\end{document}
