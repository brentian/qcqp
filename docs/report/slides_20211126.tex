\documentclass[aspectratio=1610]{beamer}

\usepackage[font=small,labelfont=bf]{caption}
\usepackage{longtable}
\usepackage{subfiles}
\usepackage{subfig}
\usepackage{booktabs}
\setlength{\tabcolsep}{6pt}
\usepackage{xcolor}
\usepackage{mathrsfs}
\usepackage{ulem}
\title{A QCQP Solver}
\author{Chuwen Zhang}
%%%%%%%%%%%%%%%%
% start my commands
%%%%%%%%%%%%%%%%

\newcommand{\lm}{\lambda_\textbf{max}}
\newcommand{\trace}{\textbf{trace}}
\newcommand{\diag}{\textbf{diag}}
\newcommand{\rank}{\textbf{rank}}
\newcommand{\model}[1]{(\textbf{#1})}
\newcommand{\mx}{\mathbf{\max}\;}
\newcommand{\mn}{\mathbf{\min}\;}
\newcommand{\st}{\mathrm{s.t.\;}}
\newcommand{\ex}{\mathbf E}
\newcommand{\dx}{\;\bm dx}
\newcommand{\pr}{\mathbf P}
\newcommand{\id}{\mathbf I}
\newcommand{\bp}{\mathbb P}
\newcommand{\be}{\mathbb E}
\newcommand{\bi}{\mathbb I}
\newcommand{\bxi}{{\bm \xi}}
\newcommand{\va}{\mathbf{Var}}
\newcommand{\dif}{\mathbf{d}}
\newcommand{\minp}[2]{\min\{#1, #2\}}
\newcommand{\intp}{\mathbf{int}}
\newcommand{\apex}{\mathbf{apex}}
\newcommand{\conv}{\mathbf{conv}}
\newcommand{\red}[1]{\textcolor{red}{#1}}
\newcommand{\redsf}[1]{\textcolor{red}{\textsf{#1}}}
\newcommand{\real}{\mathbb{R}}

%%%%%%%%%%%%%%%%
% finish my commands
%%%%%%%%%%%%%%%%
% theorem environments
\newtheorem{thm}{Theorem}[section]
\newtheorem{defn}[thm]{Definition}
\newtheorem{prop}[thm]{Proposition}
\newtheorem{cor}[thm]{Corollary}
\newtheorem{lem}[thm]{Lemma}
\newtheorem{pf}[thm]{Proof}
\newtheorem{remark}[thm]{Remark}
% my code style
%%%%%%%%%%%%%%%%
% start my commands
%%%%%%%%%%%%%%%%

\usepackage{subfig}
\usepackage[style=authoryear]{biblatex}
\usepackage{bm}

\date{\today}
\addbibresource{headers/qcqp.bib}

\subtitle{Strengthen SOCP relaxation}
\begin{document}
\fontsize{9pt}{11}\selectfont
\frame{\titlepage}
\begin{frame}{QCQP}
  Recall QCQP:

  \begin{equation}
    \begin{aligned}
      \mathrm{Maximize}\quad & x^TQx + q^T x                                   \\
      \mathrm{s.t.}  \quad   & x^{T} A_i x  + a_i^Tx   \; (\le, =, \ge) \; b_i \\
                             & 0\le x\le e
    \end{aligned}
  \end{equation}

  \begin{itemize}
    \item \(Q, A_i\) maybe indefinite
  \end{itemize}
\end{frame}
\begin{frame}{Related Literature}
  \textbf{S-free sets}\(P = \{x: x \in R - int(S), S \text{ convex}\}\)
  \begin{itemize}
    \item[\xmark] Gonzalo Muñoz, '20, '21, IPCO, Maximal quadratic free sets by LP (extreme rays)
    \item Bienstock, '14, SIOPT, Cutting planes for optimization over convex function over nonconvex sets. \textbf{also lifting}
  \end{itemize}
  \textbf{SOCPr and aggregation (SOCP-representable)}
  For convex inequalities \(\{f_i \le 0\}\), consider aggregation \(\{\sum_i\lambda_i f_i \le 0 \}\)
  \begin{itemize}
    \item Burer '17, Modaresi, Vielma' 17 two quadratic functions
    \item Dey' 20, quadratic and polytope. app. for bilinear: Dey et al' 19, New SOCP relaxation and branching rule for bipartite bilinear programs
    \item Dey et al' 21, 3 quadratic intersection.
  \end{itemize}
  \textbf{Linear disjunction}
\end{frame}
\begin{frame}{Burer '17}
  For intersection with a paraboloid and quadratic function (indefinite)
  \begin{align*}
    \mathcal Y         = \quad  & \left\{y=\left(\begin{array}{c}
        \tilde{y} \\
        y_{n}
      \end{array}\right) \in R ^{n}: \begin{array}{c}
      \tilde{y}^{T} \tilde{y} \leq y_{n} \\
      \tilde{y}^{T} \tilde{Q} \tilde{y}+2 g^{T} y+f \leq 0
    \end{array}\right\}                      \\
    \textrm{Homogenize: } \quad & x = [\tilde y, y_n, x_{n+1}]                                                                                       \\
                                & A_{0}:=\begin{bmatrix}
      I     & 0            & 0            \\
      0^{T} & 0            & -\frac{1}{2} \\
      0^{T} & -\frac{1}{2} & 0
    \end{bmatrix}, \quad A_{1}:=\begin{bmatrix}
      \tilde{Q}     & 0     & \tilde{g} \\
      0^{T}         & 0     & g_{n}     \\
      \tilde{g}^{T} & g_{n} & f
    \end{bmatrix}, \quad H^{1}:=\left\{x: x_{n+1}=1\right\} \\
    \textrm{Def} \quad          & A_{t}:=\begin{bmatrix}
      (1-t) I+t \tilde{Q} & 0                                      & t \tilde{g}                            \\
      0^{T}               & 0                                      & (1-t)\left(-\frac{1}{2}\right)+t g_{n} \\
      t \tilde{g}^{T}     & (1-t)\left(-\frac{1}{2}\right)+t g_{n} & t f
    \end{bmatrix}                                                                                   \\
                                & t = 1 /(1-\lambda), \lambda = \min(\lambda(Q)) < 0                                                                 \\
    \mathcal Y_t = \quad        & \left\{y : \tilde{y}^{T} \tilde{y} \leq y_{n}, x^TA_t x \le 0 \right\}                                             \\
    \Rightarrow \quad           & \mathcal Y_t = \textrm{cl.convex.hull}( \mathcal Y)
  \end{align*}
\end{frame}
\begin{frame}{Cutting planes for QCQP}
  \begin{itemize}
    \item use SOCPr to improve QCQP relaxations
    \item consider cases for box constraint (so as to incorporate Branch and Cut)
  \end{itemize}
\end{frame}
\end{document}