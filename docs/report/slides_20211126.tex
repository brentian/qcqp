\documentclass[aspectratio=1610, 10pt]{beamer}
\usepackage[english]{babel}
\usepackage{amsmath, amsthm, subfiles, bm, hyperref, graphicx}
% my cmd
%%%%%%%%%%%%%%%%
% start my commands
%%%%%%%%%%%%%%%%
\newcommand{\lm}{\lambda_\textrm{max}}
\newcommand{\trace}{\mathbf{trace}}
\newcommand{\diag}{\mathbf{diag}}
\newcommand{\model}[1]{(\textbf{#1})}
\newcommand{\mx}{\mathbf{\max}\;}
\newcommand{\mn}{\mathbf{\min}\;}
\newcommand{\st}{\mathrm{s.t.\;}}
\newcommand{\ex}{\mathbf E}
\newcommand{\dx}{\;\bm dx}
\newcommand{\pr}{\mathbf P}
\newcommand{\id}{\mathbf I}
\newcommand{\bp}{\mathbb P}
\newcommand{\be}{\mathbb E}
\newcommand{\bi}{\mathbb I}
\newcommand{\bxi}{{\bm \xi}}
\newcommand{\va}{\mathbf{Var}}
\newcommand{\dif}{\mathbf{d}}
\newcommand{\minp}[2]{\min\{#1, #2\}}
\newcommand{\intp}{\mathbf{int}}
\newcommand{\apex}{\mathbf{apex}}
\newcommand{\conv}{\mathbf{conv}}
%%%%%%%%%%%%%%%%
% finish my commands
%%%%%%%%%%%%%%%%

% theorem environments
\setbeamertemplate{footline}[frame number]


\makeatletter
\patchcmd{\beamer@sectionintoc}
{\ifnum\beamer@tempcount>0}
{\ifnum\beamer@tempcount>-1}
{}
{}
\beamer@tocsectionnumber=0
\makeatother

% set the toc to be using the number
\setbeamertemplate{section in toc}[sections numbered]
\setbeamertemplate{subsection in toc}[subsections numbered]

\hypersetup{pdfstartview={Fit}}

\AtBeginSection[]
{
    \begin{frame}
        \frametitle{Table of Contents}
        \tableofcontents[        currentsection,         currentsubsection,         subsectionstyle=show/shaded/hide,]
    \end{frame}
}


\defbeamertemplate{section page}{mine}[1][]{%
    \begin{centering}
        \vskip1em\par
        \begin{beamercolorbox}[sep=12pt, center]{part title}
            \usebeamerfont{section title}\insertsection\par
        \end{beamercolorbox}
    \end{centering}
}

\defbeamertemplate{subsection page}{mine}[1][]
{
    \begingroup
    \begin{beamercolorbox}[sep=8pt, center]{subsection title}
        \usebeamerfont{subsection title}\insertsubsection\par
    \end{beamercolorbox}
    \endgroup
}


\setbeamertemplate{section page}[mine]
\setbeamertemplate{subsection page}[mine]

\begin{document}


\title{A Review on SOCP Relaxations for QCQP}

\author{
  Chuwen Zhang
}


\date{November 26, 2021}

\maketitle

\section{Introduction and Motivation}

\begin{frame}{The QCQP}
  We consider the QCQP,
  \begin{equation}
    \begin{aligned}
      \model{HQCQP} \quad \mx \quad & x^T Qx                                  \\
      \text{s.t. } \quad            & x^T A_i x \hspace{0.27em} (\le, =, \ge)
      \hspace{0.27em} b_i, \forall i = 1, \ldots, m                           \\
                                    & 0 \le x \le 1
    \end{aligned}
  \end{equation}
  and inhomogeneous QCQP,
  \begin{equation}
    \label{eq:inhoqcqp}
    \begin{aligned}
      \model{QCQP} \quad \mx \quad & x^T Qx + q^T x                                    \\
      \textrm{s.t.} \quad          & x^T A_i x + a_i^T x \hspace{0.27em} (\le, =, \ge)
      \hspace{0.27em} b_i                                                              \\
                                   & 0 \le x \le 1
    \end{aligned}
  \end{equation}
\end{frame}

\begin{frame}{Dilemma of SOCP relaxations}
  Scale every indefinite inequalities.
  \begin{equation}\label{eq:primal.scale}
    \begin{aligned}
      \mx & z                                                      \\
      \st & z - x^TQx \le 0                                        \\
          & x^T(A_i) x+a_i^T x \leqslant b_i \cdot s,i= 1,\dots, m \\
    \end{aligned}
  \end{equation}
  In the above case, let \(t_i = \lambda_{\max} (Q_i)\), convexify by maximum eigenvalue. Call this \textbf{SCALE}
  \begin{align*}\label{eq.scale}
    z + x^T(t_i I_n - Q_i x) \le t_i s \\
    \|x\|^2= s
  \end{align*}
  \begin{enumerate}
    \item Relaxation bounds are worse than SDR
    \item Cannot produce feasible solutions along the way
    \item Need more cutting planes
  \end{enumerate}
\end{frame}

\begin{frame}{Recent results related to QP}

  \begin{enumerate}
    \item \textbf{Aggregation} Second-order Cone Representable (SOCPr) sets and hidden convexity. \cite{ben-tal_hidden_2014}.
    \item \textbf{Aggregation} Aggregation based convexification. For two quadratic functions, \cite{burer_how_2017}, \cite{modaresi_convex_2017}. For three quadratic functions \cite{dey_obtaining_2021}
    \item \textbf{Aggregation} Intersection of a quadratic set and polytope. \cite{santana_convex_2020}
    \item \textbf{Maximal-S-Free} Maximal S-free sets, \cite{conforti_cut-generating_2015}, \cite{michalka_cutting_2013}. Note \cite{michalka_cutting_2013} also discuss how to lift first-order cut planes for general qudratic function over nonconvex sets.
    \item \(\dagger\)Simultaneous Diagonalization ..., \cite{jiang_novel_2016}, \cite{wang_new_2021}, \cite{wang_geometric_2020}
    \item \(\dagger\)Disjunctive programming
  \end{enumerate}
  \

  \(\dagger\) 5, 6 not yet covered. \
  1, 2 can be used to tighten convex relaxation.\
  4. use to generate cuts. \
  3. to be investigated.

\end{frame}

\begin{frame}[allowframebreaks]{Aggregation: idea}
  \begin{definition}
    \textbf{(Aggregation of inequalities)} Given a set of inequalities
    \begin{align*}
      X = \{x : f_i (x) \le 0, i = 1, \ldots, m\}
    \end{align*}
    An aggregation by \(\lambda \in \mathbb R^m_+\) is defined as,
    \begin{equation}
      X(\lambda) = \left\{x: \sum_i \lambda_i \cdot f_i (x) \le 0\right\}
    \end{equation}

  \end{definition}
  \begin{definition}
    \textbf{(SOCPr sets)} A cone \(F_+\) is SOCr if it can be expressed as \(F^{+}=\left\{x:\left\|B^{T} x\right\| \leq b^{T} x\right\}\), where nonzeros columns in \(B\) are independent and \(b\notin range(B)\)
  \end{definition}

  \textbf{Some Illustration}
  \textbf{(Three quadratic functions) Example by \cite{dey_obtaining_2021} and constructed convex hull}
  \includegraphics[width=\textwidth]{../figs/dey-21-3q.png}

\end{frame}
\begin{frame}{Aggregation for two inequalities}
  Consider intersection of two quadratic function.
  \begin{align*}
     & F_0 = \{x : x^T A_0 x \le 0\}, F_1 = \{x : x^T A_1 x \le 0\} \\
     & F_s = \{x : x^T A_s x \le 0\}, A_s = sA_1 + (1 - s) A_0
  \end{align*}
  \begin{itemize}
    \item Suppose \(F_0\) is SOCr, \(F_1\) is a general quadratic set.
    \item (2017) \cite{burer_how_2017} shows a unified approach to convexify intersection, similar results in \cite{modaresi_convex_2017}, useful in Trust-region subproblems. problems like \(\|x\|^2 \le s\)
    \item (2020) \cite{santana_convex_2020} consider one Q-constraint over a polytope, applied in a bilinear bipartite graph problem.
    \item (2021) \cite{dey_obtaining_2021} shows conditions for existence of aggregation for three quadratic constraints.
    \item No easy answer to intersection of arbitrary number of Q-constraints
  \end{itemize}
\end{frame}

\begin{frame}[allowframebreaks]{Some potential improvements to SOCP relaxations}
  \textbf{Problem}: \textbf{SCALE} method is actually producing convex hull of,
  \begin{align*}
    x^TQ_i x + q_i x + b_i \le 0 \\
    \|x\|^2 \le s
  \end{align*}
  verified by \cite{burer_how_2017}, but we have to estimate \(s\).

  \textbf{Workaround}: use box \(x \in [l, u]\) to replace \(\|x\|^2 \le s\), for any \(\Sigma \ge 0\)
  \begin{equation}
    (x-l)^T\Sigma (x-u) \le 0
  \end{equation}
  When \(\Sigma \succeq 0\), it is an ellipsoid covering the box.
\end{frame}


\section{Maximal S-Free Sets and Cutting Planes}
\begin{frame}{Maximal S-Free and intersection cuts}
  Linear relaxation and quadratic free, \cite{chmiela_implementation_2021}, illustration

  \includegraphics{../figs/maximal-s-free.png}

\end{frame}
\begin{frame}{Maximal S-Free and lifting}
  Lift a first order underestimation,

  Consider \(x^TQx\) convex and minimize over \(R^n - \intp(P)\), \(P\) is a good convex set like polytope.
  \begin{align*}
    z                                          & \ge x^TQx                                                               \\
                                               & \ge 2y^TQ(x-y) + y^TQy, \forall y                                       \\
    \textbf{Lifting: } \dagger \alpha, p \quad & \ge (2y^TQ + \textcolor{red}{\alpha \cdot p}^T)(x-y) + y^TQy, \forall y \\
  \end{align*}
  Get \(\alpha, p\) by solving an optimization problem. \(\dagger \) is feasible for all \(y \notin P\) \textbf{LFO}
\end{frame}

\begin{frame}{Future Work}
  \begin{enumerate}
    \item Implement the workaround for \(\|x\|^2 \le s\)
    \item In \textbf{LFO} and \textbf{convexification}, both has to figure out a proper convex set \(P\).
    \item Summarize the readings in disjunctive and simultaneous diagonalization.
  \end{enumerate}
\end{frame}
\begin{frame}[allowframebreaks]{Bibliography}
  \bibliography{headers/qcqp}
  \bibliographystyle{alpha}
\end{frame}
\end{document}
