\documentclass[aspectratio=1610]{beamer}

\usepackage[font=small,labelfont=bf]{caption}
\usepackage{longtable}
\usepackage{subfiles}
\usepackage{subfig}
\usepackage{booktabs}
\setlength{\tabcolsep}{6pt}
\usepackage{xcolor}
\usepackage{mathrsfs}
\usepackage{ulem}
\title{A QCQP Solver}
\author{Chuwen Zhang}
\usepackage{subfig}
\usepackage[style=authoryear]{biblatex}
\usepackage{bm}
\title{Global QCQP Solver}
\author{Chuwen}
\date{\today}
\addbibresource{headers/qcqp.bib}


\begin{document}
\fontsize{9pt}{11}\selectfont
\frame{\titlepage}



\begin{frame}{QCQP}

  Suppose we readily have a sparse pattern \(G(V, E)\) (and extension \(E\subseteq F\)) for \(Q\)

  We denote aggregated pattern if \(G(V, E)\) stands for all data matrices.

  \(F = \bigcup_r C_r\), where \(C_r\) the maximal cliques can be computed efficiently for chordal graph \(F\).

  SD constraint \(Y \succeq 0\) is equivalent to,

  \[Y_{(i, i) \in C_r} \succeq 0, \quad \forall r\]

  Can be expressed as follows,

  \begin{equation}
    \begin{aligned}
       & E_r \in \mathbf R^{|C_r|\times n}     \\
       & (E_r)_{i,j}=\begin{cases}
        1,  \text { if } C_r(i)=j \\
        0,  \text { otherwise }
      \end{cases} \\
       & Y_r\equiv E_r Y E_r^T \succeq 0       \\
    \end{aligned}
  \end{equation}
\end{frame}

\begin{frame}{Sparse-SDP: SSDP}
  In QCQP, we have the SD constraint:
  \[Y - xx^T \succeq 0\]
  then any subprinciple matrix is also semidefinite,
  \begin{equation}
    \begin{aligned}
       & E_r (Y - x x^T) E_r^T \succeq 0
    \end{aligned}
  \end{equation}

  So we have the QCQP with \(r\) smaller SDP constraints.
  \begin{equation}\label{eq:sdp.blocks}
    \begin{aligned}
      \mathrm{Maximize}\quad & Q \bullet Y + q^Tx                               \\
      \mathrm{s.t.} \quad    & \begin{bmatrix}E_r Y E_r^T & E_rx \\ (E_rx)^T & 1\end{bmatrix}   \succeq 0, \forall r \\
    \end{aligned}
  \end{equation}

  This formulation incorporates a lot of linear constraints for \(C_r \cap C_{r'}\)
\end{frame}
\begin{frame}
  Consider the case for 2-blocks (2 cliques), let \(D_1 = C_1 \bigcap C_2\)

  \begin{equation}
    Y =  \sum_{r=1}^2 E_r^TY_r Er - E(D_1)^TY_{D_1}E(D_1)
  \end{equation}

  \begin{equation}
    \begin{aligned}
      \mathrm{Maximize}\quad & Q \bullet Y + q^Tx                               \\
      \mathrm{s.t.} \quad    & \begin{bmatrix}E_r Y E_r^T & E_rx \\ (E_rx)^T & 1\end{bmatrix}   \succeq 0, \forall r \\
    \end{aligned}
  \end{equation}

  We see the above is equivalent to \eqref{eq:sdp.blocks}, if there is no intersections for blocks, i.e., \(D_1 = \emptyset\),
  it can be further simplified by letting \(Q_r = E_r Q E_r^T\)
  \begin{equation}
    \begin{aligned}
      \mathrm{Maximize}\quad & \sum_{r=1}^L Q_r \bullet Y_r                     \\
      \mathrm{s.t.} \quad    & \begin{bmatrix}Y_r & E_rx \\ (E_rx)^T & 1\end{bmatrix}   \succeq 0, \forall r \\
                             & \mathrm{diag}(Y_r) \le E_r x
    \end{aligned}
  \end{equation}

  The same for any similarity transformations
\end{frame}
% \begin{frame}{Warm-start}
%   \begin{equation}
%     \begin{aligned}
%       \min_{\lambda, \mu, y, Y}\quad & \lambda^Tb + \mu^Te + 1                    \\
%       \mathrm{s.t.} \quad            & Y = Q + \sum_i \lambda_i A_i               \\
%                                      & q + \sum_i \lambda_i a_i + \mu - 2 y \le 0 \\
%                                      & Y \succeq yy^T                             \\
%                                      & \lambda, \mu \ge 0
%     \end{aligned}
%   \end{equation}
% \end{frame}
\begin{frame}
  \scriptsize
  \printbibliography
\end{frame}

\end{document}