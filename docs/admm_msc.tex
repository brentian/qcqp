\documentclass[a4paper, 10pt]{article}
\usepackage[english]{babel}
\usepackage{amsmath, amsthm, mathrsfs, subfiles, bm, hyperref, graphicx}
\usepackage{caption, longtable, booktabs}
\usepackage{cancel}
\usepackage[top=2cm, bottom=4.5cm, left=3cm, right=3.5cm]{geometry}
% my cmd
%%%%%%%%%%%%%%%%
% start my commands
%%%%%%%%%%%%%%%%
\newcommand{\lm}{\lambda_\textrm{max}}
\newcommand{\trace}{\mathbf{trace}}
\newcommand{\diag}{\mathbf{diag}}
\newcommand{\model}[1]{(\textbf{#1})}
\newcommand{\mx}{\mathbf{\max}\;}
\newcommand{\mn}{\mathbf{\min}\;}
\newcommand{\st}{\mathrm{s.t.\;}}
\newcommand{\ex}{\mathbf E}
\newcommand{\dx}{\;\bm dx}
\newcommand{\pr}{\mathbf P}
\newcommand{\id}{\mathbf I}
\newcommand{\bp}{\mathbb P}
\newcommand{\be}{\mathbb E}
\newcommand{\bi}{\mathbb I}
\newcommand{\bxi}{{\bm \xi}}
\newcommand{\va}{\mathbf{Var}}
\newcommand{\dif}{\mathbf{d}}
\newcommand{\minp}[2]{\min\{#1, #2\}}
\newcommand{\intp}{\mathbf{int}}
\newcommand{\apex}{\mathbf{apex}}
\newcommand{\conv}{\mathbf{conv}}
\newcommand{\red}[1]{\textcolor{red}{#1}}
\title{ADMM for QCQP}
\author{Chuwen Zhang}
\date{\today}


\usepackage{subfiles}
\usepackage{graphicx} 
\usepackage{subfiles} 
\usepackage{subfig}
\usepackage[table]{xcolor}
\usepackage[font=small,labelfont=bf]{caption}

\newtheorem{theorem}[]{Theorem}


\begin{document}
\maketitle
% {
%     \setcounter{tocdepth}{3}
%     \tableofcontents
% }
\section{MSC}

Consider QCQP,
\begin{equation}
    \label{eq:inhoqcqp}
    \begin{aligned}
        \model{QCQP} \quad \mx \quad & x^T Qx + q^T x                           \\
        \textrm{s.t.} \quad          & x^T A_i x + a_i^T x \le b_i, i=1, ..., m \\
    \end{aligned}
\end{equation}
In addition, we may have an extra regularity inequality by a \(\|x\| \le \delta\), box \(x \in [0, 1]^n\), or an ellipsoid.


Let basis \(V\) whose columns are \([v_1, ..., v_r]\) convexify the objective and quadratic inequalities, where,
\begin{align*}
    - Q & +  V\Lambda_0 V^T \succeq 0 \\
    A_1 & +  V\Lambda_1 V^T \succeq 0 \\
    ...                               \\
    A_j & +  V\Lambda_j V^T \succeq 0 \\
\end{align*}

We can introduce the MSC relaxation,
\begin{align}
    \model{MSC} \quad \mx \quad & \diag(\Lambda_0)^Ty - x^T\left(Q + \diag(\Lambda_0)\right)x \\
    \mathrm{s.t.} \quad         & (y, x) \in \Omega                                           \\
                                & (v_i^Tx)^2 \le y_i
\end{align}


\section{ADMM}

With \(y\) being an overestimate of \(V^Tx \circ V^Tx\), consider the following system,

\begin{align*}
    ( \mu_i)    \quad & \xi_i \cdot (v_i^Tx) = y_i           \\
                      & \xi_i^2 \le y_i,  (v_i^Tx)^2 \le y_i
\end{align*}


The AL,

\begin{equation}
    \mathcal L = ... + \sum_{i=1}^r \mu_i \left (\xi_i \cdot(v_i^Tx) - y_i\right) + \frac{\rho}{2}\sum_{i=1}^r \left(\xi_i \cdot (v_i^Tx) - y_i\right)^2
\end{equation}

\subsection{An ADMM method}
At iteration \(k\),

\begin{subequations}
    \begin{align}
        \label{eq.admm.iterx}   (x^{k+1}, y^{k+1}) & = \arg\min_{x, y} L\left(x,y,\xi^k,\mu^k\right)                         \\
        \label{eq.admm.iterxi}   \xi^{k+1}         & = \arg\min_{\xi} L\left(x^{k+1}, y^{k+1}, \mu^k\right)                  \\
        \mu^{k+1}                                  & = \mu^k + \tau^k\rho\left((\xi^{k+1})^T V^T x^{k+1} - e^Ty^{k+1}\right)
    \end{align}
\end{subequations}
The optimization problem \eqref{eq.admm.iterx} is convex that can be solved effectively. The iteration for \(\xi\) can be decomposed into \(r\) subproblems, at iteration \(k+1\), for \(i = 1, ..., r\),

\begin{equation}
    \begin{aligned}
        \mn & \mu_i^k \left (\xi_i \cdot(v_i^Tx^{k+1}) - y_i^{k+1}\right) + \frac{\rho}{2} \left(\xi_i \cdot (v_i^Tx^{k+1}) - y_i^{k+1}\right)^2 \\
        \st & \xi_i^2 \le y_i^{k+1}
    \end{aligned}
\end{equation}

which is a one dimension problem that allows analytic solution.

\begin{theorem}
    The termination criterion for ADMM,
    \begin{enumerate}
        \item Primal residual. \(\left \|(V^Tx^{k+1}) \circ \xi^{k+1} - y^{k+1} \right\| \le \epsilon\)
        \item Dual residual. \(\left\|(V^Tx^{k+1}) \circ (\xi^{k+1}-\xi^k)\right\| \le \epsilon\)
    \end{enumerate}
\end{theorem}


% \bibliography{headers/qcqp}
% \bibliographystyle{apalike}
\section{Review}
\end{document}
