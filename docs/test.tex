\documentclass[a4paper, 10pt]{article}
\usepackage[english]{babel}
\usepackage{amsmath, amsthm, subfiles, bm, hyperref, graphicx}
\usepackage{caption, longtable, booktabs}
\usepackage{cancel}
\usepackage[top=2cm, bottom=4.5cm, left=3cm, right=3.5cm]{geometry}
% my cmd
%%%%%%%%%%%%%%%%
% start my commands
%%%%%%%%%%%%%%%%
\newcommand{\lm}{\lambda_\textrm{max}}
\newcommand{\trace}{\mathbf{trace}}
\newcommand{\diag}{\mathbf{diag}}
\newcommand{\model}[1]{(\textbf{#1})}
\newcommand{\mx}{\mathbf{\max}\;}
\newcommand{\mn}{\mathbf{\min}\;}
\newcommand{\st}{\mathrm{s.t.\;}}
\newcommand{\ex}{\mathbf E}
\newcommand{\dx}{\;\bm dx}
\newcommand{\pr}{\mathbf P}
\newcommand{\id}{\mathbf I}
\newcommand{\bp}{\mathbb P}
\newcommand{\be}{\mathbb E}
\newcommand{\bi}{\mathbb I}
\newcommand{\bxi}{{\bm \xi}}
\newcommand{\va}{\mathbf{Var}}
\newcommand{\dif}{\mathbf{d}}
\newcommand{\minp}[2]{\min\{#1, #2\}}
\newcommand{\intp}{\mathbf{int}}
\newcommand{\apex}{\mathbf{apex}}
\newcommand{\conv}{\mathbf{conv}}
\newcommand{\red}[1]{\textcolor{red}{#1}}
\title{QCQP: The Low-Rank Case}
\author{Chuwen Zhang}
\date{\today}


\usepackage{subfiles}
\usepackage{graphicx} 
\usepackage{subfiles} 
\usepackage{subfig}
\usepackage[table]{xcolor}
\usepackage[font=small,labelfont=bf]{caption}



\begin{document}
\maketitle
% {
%     \setcounter{tocdepth}{3}
%     \tableofcontents
% }
\section{The Low-Rank QCQP}

Consider QCQP,
\begin{equation}
    \label{eq:inhoqcqp}
    \begin{aligned}
        \model{QCQP} \quad \mx \quad & x^T Qx + q^T x                           \\
        \textrm{s.t.} \quad          & x^T A_i x + a_i^T x \le b_i, i=1, ..., m \\
    \end{aligned}
\end{equation}
In addition, we may have \(\|x\| \le \delta\) or box \(x \in [0, 1]^n\).

\paragraph{Rank-\(r\) indefiniteness}

Formally, a quadratic inequality induced by a symmetric matrix \(A\) can be expressed as the following,

\begin{equation}
    x^T \left(\sum_{j\in J_+} \lambda_j v_jv_j^T\right)x +a^Tx \le b + \left(\sum_{j\in J_-} \lambda_j v_jv_j^T\right) \\
\end{equation}

where \(\{\lambda_j\}_{j=1}^n\) are the absolute value of the eigenvalues of \(A\), \(\{v_j\}_{j=1}^n\) are corresponding eigenvectors, and \(J_-, J_+\) are index sets of negative and non-negative eigenvalues. If A is rank-\(r\) indefinite, then \(|J_-| = r\).

For clarity, we say a quadratic inequality by \(A_i\) is \emph{rank-\(r\) indefinite} if first \(r\) eigenvalues are negative. In comparison, a maximization problem with matrix \(Q\) is rank-\(r\) indefinite if last \(r\) eigenvalues are nonnegative.

\section{Rank-\(r\) Indefinite}

Consider the case where \(r=1\).
\subsection{Homogeneous quadratic inequality}
Consider a homogeneous quadratic inequality with matrix \(Q\),
\begin{equation}
    \Omega = \{x: x^TQx \le 0\}
\end{equation}


Note if \(r = 0\), \( \Omega\) is readily a convex set. If there is only one negative \(\lambda_j\), then it is \emph{second-order cone representable} (SOCr):

\begin{equation}
    \Omega = \{x: \|R^Tx\| \le \sqrt {-\lambda_j} v_j^Tx \} \cup \{x: \|R^Tx\| \le - \sqrt {-\lambda_j} v_j^Tx \}
\end{equation}

Then we can decompose the problem into two disjunctions, each of which is convex.


\subsection{Inhomogeneous quadratic inequality}
\subsection{Rank-\(1\) indefinite function}

If \(m = 0\), that is, there is no further constraints, then \(x^TQx = x^T(\lambda\cdot vv^T-RR^T)x + q^Tx\), \eqref{eq:inhoqcqp} can be rewritten into the following form,

\begin{equation}\label{eq:qp_unc_r1}
    \begin{aligned}
        \mx \quad & z                                              \\
        \st \quad & z + x^TRR^Tx - q^Tx \le \lambda \cdot (v^Tx)^2
    \end{aligned}
\end{equation}

then,
\begin{equation}\label{eq:qp_unc_r1_conic}
    \begin{aligned}
        \mx \quad & z                                                         \\
        \st \quad & \left\|R^Tx \right \|^2 \le \lambda \cdot \rho + q^Tx - z \\
                  & (v^Tx)^2 \le \rho                                         \\
    \end{aligned}
\end{equation}
Obviously, \eqref{eq:qp_unc_r1_conic} is a convex relaxation of \eqref{eq:qp_unc_r1}. The problem is unbounded without regularization on \((\rho, v^Tx)\) that can be constructed from the boundary of \(x\). In practice, such boundary could be a box or a ball by \(2\)-norm.

\subsection{Rank-\(r\) indefinite}
Now we consider the case for \(r \ge 2\), the convex relaxation,

\begin{equation}\label{eq:qp_unc_rr_conic}
    \begin{aligned}
        \mx \quad & z                                                                                  \\
        \st \quad & \sum_{j \in J_{-}} \lambda_{j} z_j - q^T x+z \le  \sum_{j \in J_{+}} \lambda_j z_j \\
                  & y_j \ge (x_j^T v_j)^2, j= 1, ..., n
    \end{aligned}
\end{equation}

Note we have \(r\) positive eigenvalues, so \(y_j = (x_j^Tv_j)^2\) is tight for any \(j \in J_-\), if \(y_j = (x_j^Tv_j)^2\) are active for rest of \(j\in J_+\), the problem is solved.

\eqref{eq:qp_unc_rr_conic} is likely to be unbounded, if \(r \ge 1\) and no regularity constraint such a box, ball, or an ellipsoid is present.



\pagebreak

\section{Computational Results}
\subsection{Boxed Constrained QP with rank-\(r\) indefinite Q}
\scriptsize
\begin{table}[h!]
    \begin{tabular}{llrrrrrl}
        \toprule
        {} & prob\_num & solve\_time & best\_bound & best\_obj & node\_time & nodes    & method        \\
        \midrule
        0  & 12:0:0    & 0.134       & 258.6487    & 258.6487  & 0.000      & 726.0    & grb           \\
        1  & 12:0:0    & 0.040       & 258.6487    & 258.6487  & 0.025      & 1.0      & bb\_sdp       \\
        2  & 12:0:0    & 0.174       & 258.6488    & 258.6487  & 0.001      & 15.0     & bb\_msc       \\
        3  & 12:0:0    & 15.395      & 258.6497    & 258.6485  & 0.003      & 1623.0   & bb\_nsocp     \\
        \midrule
        0  & 20:0:0    & 100.025     & 47.1953     & 47.1886   & 0.000      & 166519.0 & grb           \\
        1  & 20:0:0    & \red{0.427} & 47.1894     & 47.1886   & 0.002      & 21.0     & \red{bb\_msc} \\
        2  & 20:0:0    & 100.061     & 47.2924     & 46.9227   & 0.011      & 3001     & bb\_sdp       \\
        \midrule
        0  & 100:0:0   & 100.024     & 4966.1144   & 0.1682    & 0.000      & 7.0      & grb           \\
        1  & 100:0:0   & 100.043     & 109.0224    & -84.1555  & 0.073      & 1429.0   & bb\_sdp       \\
        2  & 100:0:0   & 1.609       & 0.1705      & 0.1682    & 0.026      & 17.0     & bb\_msc       \\
        \midrule
        0  & 200:0:0   & 100.166     & 71975.3347  & 0.0743    & 0.000      & 1.0      & grb           \\
        1  & 200:0:0   & 4.228       & 0.0776      & 0.0743    & 0.118      & 15.0     & bb\_msc       \\
        \bottomrule
    \end{tabular}
    \caption{\(r=1\)}

\end{table}
\begin{table}[h!]
    \begin{tabular}{llrrrrrl}
        \toprule
        {} & prob\_num & solve\_time & best\_bound & best\_obj & node\_time & nodes & method  \\
        \midrule
        0  & 100:0:0   & 100.019     & 5013.7126   & 0.2170    & 0.000      & 7.0   & grb     \\
        1  & 100:0:0   & 7.859       & 0.2195      & 0.2166    & 0.027      & 87.0  & bb\_msc \\
        \bottomrule
    \end{tabular}
    \caption{\(r=2, n=100\)}
\end{table}
\begin{table}[h!]
    \begin{tabular}{llrrrrrl}
        \toprule
        {} & prob\_num & solve\_time & best\_bound & best\_obj & node\_time & nodes  & method  \\
        \midrule
        0  & 100:0:0   & 100.015     & 11652.5910  & 526.5704  & 0.000      & 1.0    & grb     \\
        1  & 100:0:0   & 100.073     & 1540.5022   & 309.6513  & 0.019      & 1263.0 & bb\_msc \\
        \bottomrule
    \end{tabular}
    \caption{\(r=10, n=100\)}
\end{table}
\normalsize

\subsection{Ball Constrained QP}
In this part, we test \eqref{eq:qp_unc_rr_conic} with regularity,
\begin{equation}
    y^Te = \|x\|^2 \le \delta^2
\end{equation}
\scriptsize
\begin{table}[h!]
    \begin{tabular}{llrrrrrl}
        \toprule
        {} & prob\_num & solve\_time & best\_bound  & best\_obj    & node\_time & nodes   & method    \\
        \midrule
        0  & 12:0:0    & 100.052     & 2282.8435    & 2198.0355    & 0.000      & 62831.0 & grb       \\
        1  & 12:0:0    & 0.027       & 2198.0355    & 2198.0355    & 0.017      & 1.0     & bb\_msc   \\
        2  & 12:0:0    & 0.013       & 2198.0355    & 2198.0354    & 0.002      & 2.0     & bb\_nsocp \\
        \midrule
        0  & 120:0:0   & 0.134       & 1.852151e+06 & 1.852151e+06 & 0.024      & 1       & bb\_msc   \\
        \midrule
        0  & 1200:0:0  & 7.472       & 1.874103e+09 & 1.874104e+09 & 2.994      & 1       & bb\_msc   \\
        \bottomrule
    \end{tabular}
\end{table}
\normalsize

\subsection{Box QP in \cite{luo_new_2019}}

We mark \textsf{Luo, ADMBB} as the method of using ADMM + BB using in \cite{luo_new_2019}.  Their method is basically the following:
\begin{itemize}
    \item Their ADMM penalizes \(V^Tx = Z\)
    \item The BB procedure is the same, except for the branching rules. In \textsf{Luo, ADMBB}, the pivot is calculated by \(z := \frac{1}{2}(l_z + u_z)\) instead of \(z^*\)
    \item In the Box QP, the convex relaxation is the same. They use CPLEX to solve QCP directly.
    \item The instances are generated following the tradition in \cite{le_an_solving_1997}.
\end{itemize}

\scriptsize
\begin{table}[h!]
    \centering
    \begin{tabular}{llllll}
        \toprule
        \(n\) & \(r\) & solve\_time & nodes & \#ADMM & method              \\
        \midrule
        50    & 5     & 29.089      & 11223 & -      & grb                 \\
        50    & 5     & 11.059      & 529   &        & bb\_msc             \\
        50    & 5     & 8.114       & 245   & 3      & \textsf{Luo, ADMBB} \\
        100   & 5     & 90.549      & 439   & -      & grb                 \\
        100   & 5     & 9.683       & 529   &        & bb\_msc             \\
        100   & 5     & 8.776       & 305   & 5      & \textsf{Luo, ADMBB} \\
        100   & 5     & 241.730     & 439   & -      & grb                 \\
        100   & 5     & 31.602      & 537   &        & bb\_msc             \\
        100   & 5     & 12.56       & 305   & 8      & \textsf{Luo, ADMBB} \\
        \bottomrule
    \end{tabular}
    \caption{The box QP instances generated using the method in \cite{luo_new_2019}.
        We terminate Gurobi very early here (gap \(\le e^{-3}\))}
\end{table}
\normalsize

\bibliography{headers/qcqp}
\bibliographystyle{apalike}
\section{Review}
\end{document}
